% main.tex
% This is the master document for the entire solutions book.

\documentclass[letter,12pt,oneside]{book} % Using 'book' class for chapters
                                  % Use 'article' if you don't want formal chapters

\renewcommand{\thechapter}{\Roman{chapter}} % This is the key line
% This is the key line to make sections numbered simply as 1, 2, 3...
\renewcommand{\thesection}{\arabic{section}}

% Optional: If you want subsections to be 1.1, 1.2, etc. (relative to the section, not chapter)
\renewcommand{\thesubsection}{\thesection.\arabic{subsection}}

% --- NEW: For enumerate numbering ---
% This redefines the first level of enumeration to be SectionNumber.ItemNumber
\renewcommand{\theenumi}{\thesection.\arabic{enumi}}
% This ensures the label uses the new numbering format
\usepackage{enumitem} % Recommended package for list customization
\setlist[enumerate,1]{label=\theenumi} % Apply the label format to the first level of enumerate
% --- Preamble: Global Settings and Packages (as discussed previously) ---
\usepackage[utf8]{inputenc}
\usepackage[T1]{fontenc}
\usepackage{lmodern}

\usepackage{amsmath}
\usepackage{amssymb}
\usepackage{amsthm}
\usepackage{mathtools}
\usepackage{bm}
\usepackage{geometry}
\usepackage{lipsum}
\usepackage{tcolorbox}
\tcbuselibrary{listings,breakable}
\usepackage{xcolor} % Make sure xcolor is loaded for color definitions

\usepackage{graphicx}
\graphicspath{{assets/}} % Path to global image assets
\usepackage{mathrsfs} % Add this line to your preamble for \mathscr support; tlmgr install jknapltx rsfs server-side

\usepackage[
    colorlinks=true,
    linkcolor=blue,
    urlcolor=blue,
    citecolor=green,
    pdftitle={Comprehensive Solutions to [Textbook Name]},
    pdfauthor={Your Name},
    pdfsubject={Mathematics Solutions},
    pdfkeywords={Math, Solutions, Textbook, LaTeX}
]{hyperref}

% Define a tcolorbox style named 'mysolutionbox' with only necessary options
\tcbset{
  mysolutionbox/.style={
    colback=gray!10, % Light grey background
    colframe=gray!50, % Optional: darker grey frame
    boxrule=0.5pt, % Optional: adjust frame thickness
    arc=4pt, % Optional: rounded corners
    left=5pt, right=5pt, top=5pt, bottom=5pt, % Adjust internal padding
    % Explicitly set paragraph spacing and remove indent
    before upper={%
      \setlength{\parskip}{0.6\baselineskip}% % Set specific spacing
      \setlength{\parindent}{0pt}% % No indent
      % \everypar{\relax}% % Keep this in for robustness, but try without it first
    },
    breakable, % Allow the box to break across pages
    % NO TITLE-RELATED OPTIONS HERE
  }
}

% Define the solution environment using the 'mysolutionbox' style
\newenvironment{solution}{
  \begin{tcolorbox}[mysolutionbox] % Apply the defined style here
    \textbf{Solution:}\quad % Manually add the title here, outside of tcolorbox options
}{
  \hfill$\square$\par\vspace{1ex} % Square at the end
  \end{tcolorbox}
}

% Custom theorem/solution environments
\newtheorem{theorem}{Theorem}[section]
\newtheorem{proposition}[theorem]{Proposition}
\newtheorem{lemma}[theorem]{Lemma}
\newtheorem{definition}[theorem]{Definition}
\newtheorem{example}[theorem]{Example}
\newtheorem{remark}[theorem]{Remark}
% \newenvironment{solution}{\textbf{Solution:}\quad}{\hfill$\square$\par\vspace{1ex}}

% --- Document Information ---
\title{Solutions for Selected Problems from Aluffi's \textit{Algebra: Chapter 0}} % Use \textit{} for book title
\author{}
\date{\today}

% --- Begin Document ---
\begin{document}

% --- Custom commands ---
\newcommand{\im}{\operatorname{im}}

% Title page
\maketitle

% Table of Contents
\frontmatter % Command for 'book' class to switch to Roman numbering for front matter
\tableofcontents
\mainmatter % Command for 'book' class to switch back to Arabic numbering for main content

% --- Include Chapters and their Sections ---

% Chapter 1
\chapter{Preliminaries: Set theory and categories} % This defines the chapter title and number
% Now, input each section's solutions within this chapter
\section{Naive set theory}
\begin{enumerate}
    \item Locate a discussion of Russell's paradox, and understand it.
    \item Prove that if $\sim$ is an equivalence relation on a set $S$, then the corresponding family $\mathcal{P}_{\sim}$ defined in \S1.5 is indeed a partition of $S$: that is, its elements are nonempty, disjoint, and their union is $S$. [\S1.5]

          \begin{solution}
            \lipsum[1]
          \end{solution}

    \item Given a partition $\mathcal{P}$ on a set $S$, show how to define a relation $\sim$ on $S$ such that $\mathcal{P}$ is the corresponding partition. [\S1.5]
    \item How many different equivalence relations may be defined on the set $\{1, 2, 3\}$?
    \item Give an example of a relation that is reflexive and symmetric but not transitive. What happens if you attempt to use this relation to define a partition on the set? (Hint: Thinking about the second question will help you answer the first one.)
    \item Define a relation $\sim$ on the set $\mathbb{R}$ of real numbers by setting $a \sim b \iff b-a \in \mathbb{Z}$. Prove that this is an equivalence relation, and find a `compelling' description for $\mathbb{R}/\sim$. Do the same for the relation $\approx$ on the plane $\mathbb{R} \times \mathbb{R}$ defined by declaring $(a_1, a_2) \approx (b_1, b_2) \iff b_1 - a_1 \in \mathbb{Z} \text{ and } b_2 - a_2 \in \mathbb{Z}$. [\S II.8.1, II.8.10]
\end{enumerate}

\section{Functions between sets}
\begin{enumerate}
    \item jhsdf
    \item gdfhs
\end{enumerate}
\section{Categories}
\begin{enumerate}
    \item Let $\mathsf{C}$ be a category. Consider a structure $\mathsf{C}^{\mathrm{op}}$ with
          \begin{itemize}
              \item $\mathrm{Obj}(\mathsf{C}^{\mathrm{op}}) := \mathrm{Obj}(\mathsf{C})$;
              \item for $A, B$ objects of $\mathsf{C}^{\mathrm{op}}$ (hence objects of $\mathsf{C}$), $\mathrm{Hom}_{\mathsf{C}^{\mathrm{op}}}(A, B) := \mathrm{Hom}_{\mathsf{C}}(B, A)$.
          \end{itemize}
          Show how to make this into a category (that is, define composition of morphisms in $\mathsf{C}^{\mathrm{op}}$ and verify the properties listed in \S3.1).

          Intuitively, the `opposite' category $\mathsf{C}^{\mathrm{op}}$ is simply obtained by `reversing all the arrows' in $\mathsf{C}$. [\S5.1, \S VIII.1.1, \S IX.1.2, IX.1.10]

    \item If $A$ is a finite set, how large is $\mathrm{End}_{\mathsf{Set}}(A)$?
        \begin{solution}
            By Example 3.2 and excercise 2.10 from cahpter 1, we conclude that $|\mathrm{End}_{\mathsf{Set}}(A)|=n^n$
        \end{solution}
    \item Formulate precisely what it means to say that $1_A$ is an identity with respect to composition in Example 3.3, and prove this assertion. [\S3.2]

    \item Can we define a category in the style of Example 3.3 using the relation $<$ on the set $\mathbb{Z}$?

    \item Explain in what sense Example 3.4 is an instance of the categories considered in Example 3.3. [\S3.2]

    \item (Assuming some familiarity with linear algebra.) Define a category $\mathsf{V}$ by taking $\mathrm{Obj}(\mathsf{V}) = \mathbb{N}$ and letting $\mathrm{Hom}_{\mathsf{V}}(n, m) = \text{the set of } m \times n \text{ matrices with real entries}$, for all $n, m \in \mathbb{N}$. (I will leave the reader the task of making sense of a matrix with 0 rows or columns.) Use product of matrices to define composition. Does this category `feel' familiar? [\S VI.2.1, \S VIII.1.3]

    \item Define carefully objects and morphisms in Example 3.7, and draw the diagram corresponding to composition. [\S3.2]

    \item A \textit{subcategory} $\mathsf{C}'$ of a category $\mathsf{C}$ consists of a collection of objects of $\mathsf{C}$ with sets of morphisms $\mathrm{Hom}_{\mathsf{C}'}(A, B) \subseteq \mathrm{Hom}_{\mathsf{C}}(A, B)$ for all objects $A, B$ in $\mathrm{Obj}(\mathsf{C}')$, such that identities and compositions in $\mathsf{C}$ make $\mathsf{C}'$ into a category. A subcategory $\mathsf{C}'$ is full if $\mathrm{Hom}_{\mathsf{C}'}(A, B) = \mathrm{Hom}_{\mathsf{C}}(A, B)$ for all $A, B$ in $\mathrm{Obj}(\mathsf{C}')$. Construct a category of \textit{infinite sets} and explain how it may be viewed as a full subcategory of $\mathsf{Set}$. [4.4, \S VI.1.1, \S VIII.1.3]

    \item An alternative to the notion of \textit{multiset} introduced in \S2.2 is obtained by considering sets endowed with equivalence relations; equivalent elements are taken to be multiple instances of elements `of the same kind'. Define a notion of morphism between such enhanced sets, obtaining a category $\mathsf{MSet}$ containing (a `copy' of) $\mathsf{Set}$ as a full subcategory. (There may be more than one reasonable way to do this! This is intentionally an open-ended exercise.) Which objects in $\mathsf{MSet}$ determine ordinary multisets as defined in \S2.2 and how? Spell out what a morphism of multisets would be from this point of view. (There are several natural notions of morphisms of multisets. Try to define morphisms in $\mathsf{MSet}$ so that the notion you obtain for ordinary multisets captures your intuitive understanding of these objects.) [\S2.2, \S3.2, 4.5]

    \item Since the objects of a category $\mathsf{C}$ are not (necessarily interpreted as) sets, it is not clear how to make sense of a notion of `subobject' in general, extrapolating the notion of \emph{subset}. In some situations it \textit{does} make sense to talk about subobjects, and the subobjects of any given object $A$ in $\mathsf{C}$ are in one-to-one correspondence with the morphisms $A \to \Omega$ for a fixed, special object $\Omega$ of $\mathsf{C}$, called a \textit{subobject classifier}. Show that $\mathsf{Set}$ has a subobject classifier.

    \item Draw the relevant diagrams and define composition and identities for the category $\mathsf{C}^{A,B}$ mentioned in Example 3.9. Do the same for the category $\mathsf{C}^{\alpha,\beta}$ mentioned in Example 3.10. [\S5.5, 5.12]
\end{enumerate}
\section{Morphisms}
\begin{enumerate}
    \item Composition is defined for \textit{two} morphisms. If more than two morphisms are given, e.g.,
    \[ A \xrightarrow{f} B \xrightarrow{g} C \xrightarrow{h} D \xrightarrow{i} E, \]
    then one may compose them in several ways, for example:
    \[ (ih)(gf), \quad (i(hg))f, \quad i((hg)f), \quad \text{etc.} \]
    so that at every step one is only composing two morphisms. Prove that the result of any such nested composition is independent of the placement of the parentheses. (Hint: Use induction on $n$ to show that any such choice for $f_n f_{n-1} \cdots f_1$ equals
    \[ ((\cdots((f_n f_{n-1}) f_{n-2}) \cdots) f_1). \]
    Carefully working out the case $n=5$ is helpful.) [\S4.1, \S II.1.3]

\item In Example 3.3 we have seen how to construct a category from a set endowed with a relation, provided this latter is reflexive and transitive. For what types of relations is the corresponding category a groupoid (cf. Example 4.6)? [\S4.1]

\item Let $A, B$ be objects of a category $\mathsf{C}$, and let $f \in \text{Hom}_{\mathsf{C}}(A, B)$ be a morphism.
    \begin{itemize}
        \item Prove that if $f$ has a right-inverse, then $f$ is an epimorphism.
        \item Show that the converse does not hold, by giving an explicit example of a category and an epimorphism without a right-inverse.
    \end{itemize}

\item Prove that the composition of two monomorphisms is a monomorphism. Deduce that one can define a subcategory $\mathsf{C}_{\text{mono}}$ of a category $\mathsf{C}$ by taking the objects as in $\mathsf{C}$ and defining $\text{Hom}_{\mathsf{C}_{\text{mono}}}(A, B)$ to be the subset of $\text{Hom}_{\mathsf{C}}(A, B)$ consisting of monomorphisms, for all objects $A, B$. (Cf. Exercise 3.8; of course, in general $\mathsf{C}_{\text{mono}}$ is not full in $\mathsf{C}$.) Do the same for epimorphisms. Can you define a subcategory $\mathsf{C}_{\text{nonmono}}$ of $\mathsf{C}$ by restricting to morphisms that are \textit{not} monomorphisms?

\item Give a concrete description of monomorphisms and epimorphisms in the category $\mathsf{MSet}$ you constructed in Exercise 3.9. (Your answer will depend on the notion of morphism you defined in that exercise!)
\end{enumerate}
\section{Universal properties}
\begin{enumerate}
      \item Prove that a final object in a category $\mathsf{C}$ is initial in the opposite category $\mathsf{C}^{\text{op}}$ (cf. Exercise 3.1).
      \item Prove that $\emptyset$ is the unique initial object in $\mathsf{Set}$. [\S5.1]
      \item Prove that final objects are unique up to isomorphism. [\S5.1]
      \item What are initial and final objects in the category of `pointed sets' (Example 3.8)? Are they unique?
            \begin{solution}
                  The natural candidate in this case is the pair $(\{x\}, x)$, i.e., any singleton with its unique element as the base point. Given any pointed set $(S, s)$, there is a unique function $\{x\} \to S$ such that $x \mapsto s$, and a unique function $S \to \{x\}$ such that $s \mapsto x$. Hence $(\{x\}, x)$ is both initial and final in the category of pointed sets. In either case, it is unique up to unique isomorphism.
            \end{solution}
      \item What are the final objects in the category considered in \S5.3? [\S5.3]
            \begin{solution}
                  They are the functions $f:A \to \{x\}$ to any singleton. If $a'\sim a''$ then obviously $f(a') = f(a'')$, whence $f$ is an object in the given category. And it is a final object, since for any $\varphi:A \to Z$ there is a unique function $F: Z \to \{x\}$ such that $F\circ \varphi = f$ (this function maps everything to $x$).
            \end{solution}
      \item Consider the category corresponding to endowing (as in Example 3.3) the set $\mathbb{Z}^+$ of positive integers with the divisibility relation. Thus there is exactly one morphism $d \to m$ in this category if and only if $d$ divides $m$ without remainder; there is no morphism between $d$ and $m$ otherwise. Show that this category has products and coproducts. What are their `conventional' names? [\S VII.5.1]
      \item Redo Exercise 2.9, this time using Proposition 5.4.
            \begin{solution}
                  Since we now know that the disjoint union is the coproduct in the category of sets, it immediately follows that it is well-defined up to isomorphism (as is any object defined by a universal property).
            \end{solution}
      \item Show that in every category $\mathsf{C}$ the products $A \times B$ and $B \times A$ are isomorphic, if they exist. (Hint: Observe that they both satisfy the universal property for the product of $A$ and $B$; then use Proposition 5.4.)
      \item Let $\mathsf{C}$ be a category with products. Find a reasonable candidate for the universal property that the product $A \times B \times C$ of three objects of $\mathsf{C}$ ought to satisfy, and prove that both $(A \times B) \times C$ and $A \times (B \times C)$ satisfy this universal property. Deduce that $(A \times B) \times C$ and $A \times (B \times C)$ are necessarily isomorphic.
      \item Push the envelope a little further still, and define products and coproducts for \textit{families} (i.e., indexed sets) of objects of a category.

            Do these exist in $\mathsf{Set}$?

            It is common to denote the product $\underbrace{A \times \cdots \times A}_{n \text{ times}}$ by $A^n$.

      \item Let $A$, resp. $B$ be a set, endowed with an equivalence relation $\sim_A$, resp. $\sim_B$. Define a relation $\sim$ on $A \times B$ by setting
            \[ (a_1, b_1) \sim (a_2, b_2) \iff a_1 \sim_A a_2 \text{ and } b_1 \sim_B b_2. \]
            (This is immediately seen to be an equivalence relation.)
            \begin{itemize}
                  \item Use the universal property for quotients (\S5.3) to establish that there are canonical quotient maps $q_A : A \to A/{\sim_A}$, $q_B : B \to B/{\sim_B}$, and $q : A \times B \to (A \times B)/{\sim_{A \times B}}$, and that these induce functions $(A \times B)/{\sim_{A \times B}} \to A/{\sim_A}$ and $(A \times B)/{\sim_{A \times B}} \to B/{\sim_B}$.
                  \item Prove that $(A \times B)/{\sim_{A \times B}}$, together with these induced functions, satisfies the universal property for the product of $A/{\sim_A}$ and $B/{\sim_B}$.
                  \item Conclude (without further work) that $(A \times B)/{\sim_{A \times B}} \cong (A/{\sim_A}) \times (B/{\sim_B})$.
            \end{itemize}

      \item Define the notions of \textit{fibered products} and \textit{fibered coproducts}, as terminal objects of the categories $\mathsf{C}_{\alpha,\beta}$, $\mathsf{C}^{\alpha,\beta}$ considered in Example 3.10 (cf. also Exercise 3.11), by stating carefully the corresponding universal properties.

            As it happens, $\mathsf{Set}$ has both fibered products and coproducts. Define these objects `concretely', in terms of naive set theory. [II.3.9, III.6.10, III.6.11]

            \begin{solution}
                  Let us first define the fibered product $A \times_{C} B$ of two morphisms $\alpha:A \to C$, $\beta:B \to C$ as a final object in $\mathsf{C}_{\alpha,\beta}$. The universal property is as follows: for any object $(Z,f_A,f_B)$ in $\mathsf{C}_{\alpha,\beta}$, there exists a unique morphism $Z \to A \times_C B$ such that the following diagram commutes:

                  % https://q.uiver.app/#q=WzAsNSxbMSwxLCJBXFx0aW1lc19DQiJdLFsyLDAsIkEiXSxbMiwyLCJCIl0sWzMsMSwiQyJdLFswLDEsIloiXSxbMCwxLCJcXHBpX0EiXSxbMCwyLCJcXHBpX0IiLDJdLFsxLDMsIlxcYWxwaGEiXSxbMiwzLCJcXGJldGEiLDJdLFs0LDEsImZfQSIsMCx7ImN1cnZlIjotMn1dLFs0LDIsImZfQiIsMix7ImN1cnZlIjoyfV0sWzQsMCwiXFxleGlzdHMhIiwwLHsic3R5bGUiOnsiYm9keSI6eyJuYW1lIjoiZGFzaGVkIn19fV1d
                  \[\begin{tikzcd}
                              && A \\
                              Z & {A\times_CB} && C \\
                              && B
                              \arrow["\alpha", from=1-3, to=2-4]
                              \arrow["{f_A}", curve={height=-12pt}, from=2-1, to=1-3]
                              \arrow["{\exists!}", dashed, from=2-1, to=2-2]
                              \arrow["{f_B}"', curve={height=12pt}, from=2-1, to=3-3]
                              \arrow["{\pi_A}", from=2-2, to=1-3]
                              \arrow["{\pi_B}"', from=2-2, to=3-3]
                              \arrow["\beta"', from=3-3, to=2-4]
                        \end{tikzcd}\]

                  Note that since the fibered product comes equipped with projections to $A,B$, the universal property of the product yields a map $m:A\times_C B \to A\times B$ such that $\pi_A = p_A \circ m$, $\pi_B = p_B \circ m$, where $p_A$ and $p_B$ are the projections of the product. I claim that $m$ is \emph{monic}. Indeed, assume that $m \circ f_1 = m \circ f_2$ for some morphisms $f_1,f_2:Z \to A\times_C B$. Then, applying the product projections yields $\pi_A \circ f_1 = \pi_A \circ f_2$ (call this $f_A$) and $\pi_B \circ f_1 = \pi_B \circ f_2$ (call this $f_B$). By commutativity of the fibered product square we then have $\alpha \circ f_A = \beta \circ f_B$, so $f_1 = f_2$ by the uniqueness part of the universal property. Thus, $m$ is monic.

                  Similarly, the commutative diagram for the fibered coproduct $A \sqcup_C B$ of two morphisms $\alpha:C \to A$, $\beta:C \to B$, as an initial object in the category $\mathsf{C}^{\alpha,\beta}$ is as follows (reverse all the arrows above):

                  % https://q.uiver.app/#q=WzAsNSxbMiwxLCJBXFxzcWN1cF9DQiJdLFsxLDAsIkEiXSxbMSwyLCJCIl0sWzAsMSwiQyJdLFszLDEsIloiXSxbMSwwLCJpX0EiXSxbMiwwLCJpX0IiLDJdLFszLDEsIlxcYWxwaGEiXSxbMywyLCJcXGJldGEiLDJdLFsxLDQsImZfQSIsMCx7ImN1cnZlIjotMn1dLFsyLDQsImZfQiIsMix7ImN1cnZlIjoyfV0sWzAsNCwiXFxleGlzdHMhIiwwLHsic3R5bGUiOnsiYm9keSI6eyJuYW1lIjoiZGFzaGVkIn19fV1d
                  \[\begin{tikzcd}
                              & A \\
                              C && {A\sqcup_CB} & Z \\
                              & B
                              \arrow["{i_A}", from=1-2, to=2-3]
                              \arrow["{f_A}", curve={height=-12pt}, from=1-2, to=2-4]
                              \arrow["\alpha", from=2-1, to=1-2]
                              \arrow["\beta"', from=2-1, to=3-2]
                              \arrow["{\exists!}", dashed, from=2-3, to=2-4]
                              \arrow["{i_B}"', from=3-2, to=2-3]
                              \arrow["{f_B}"', curve={height=12pt}, from=3-2, to=2-4]
                        \end{tikzcd}\]
                  In this case we get an epimorphism $p: A\sqcup B \to A\sqcup_C B$ instead.

                  Specializing now to the category $\mathsf{Set}$, we can describe the fibered product and coproduct more concretely. The monomorphism and epimorphism above translate to $A \times_C B$ being a subset of $A \times B$, and $A \sqcup_C B$ being a quotient of $A \sqcup B$. Concretely, $A \times_C B = \{(a,b) \in A \times B \mid \alpha(a) = \beta(b)\}$, while $A \sqcup_C B$ is the set of equivalence classes of the relation $\sim$ on $A \sqcup B$ generated by $\alpha(c) \sim \beta(c)$ for all $c \in C$.
            \end{solution}
\end{enumerate}

% Chapter 2
\chapter{Groups, first encounter}
\section{Definition of group}
\begin{enumerate}
    \item Write a careful proof that every group is the group of isomorphisms of a groupoid. In particular, every group is the group of automorphisms of some object in some category. [\S2.1]

    \item Consider the `sets of numbers' listed in \S1.1, and decide which are made into groups by conventional operations such as $+$ and $\cdot$. Even if the answer is negative (for example, $(\mathbb{R}, \cdot)$ is not a group), see if variations on the definition of these sets lead to groups (for example, $(\mathbb{R}^*,\cdot)$ is a group; cf. \S1.4). [\S1.2]

    \item Prove that $(gh)^{-1} = h^{-1}g^{-1}$ for all elements $g, h$ of a group $G$.

    \item Suppose that $g^2 = e$ for all elements $g$ of a group $G$; prove that $G$ is commutative.

    \item The `multiplication table' of a group is an array compiling the results of all multiplications $g \bullet h$:
          \[
              \begin{array}{|c||c|c|c|}
                  \hline
                  \bullet & e      & h           & \cdots \\
                  \hline\hline
                  e       & e      & h           & \cdots \\
                  \hline
                  g       & g      & g \bullet h & \cdots \\
                  \hline
                  \vdots  & \vdots & \vdots      & \ddots \\
                  \hline
              \end{array}
          \]
          (Here $e$ is the identity element. Of course the table depends on the order in which the elements are listed in the top row and leftmost column.) Prove that every row and every column of the multiplication table of a group contains all elements of the group exactly once (like Sudoku diagrams!).

    \item Prove that there is only one possible multiplication table for $G$ if $G$ has exactly 1, 2, or 3 elements. Analyze the possible multiplication tables for groups with exactly 4 elements, and show that there are \textit{two} distinct tables, up to reordering the elements of $G$. Use these tables to prove that all groups with $\le 4$ elements are commutative.

          (You are welcome to analyze groups with 5 elements using the same technique, but you will soon know enough about groups to be able to avoid such brute-force approaches.) [2.19]

    \item Prove Corollary 1.11.

    \item Let $G$ be a finite abelian group with exactly one element $f$ of order 2. Prove that $\prod_{g \in G} g = f$. [4.16]

    \item Let $G$ be a finite group, of order $n$, and let $m$ be the number of elements $g \in G$ of order exactly 2. Prove that $n-m$ is odd. Deduce that if $n$ is even, then $G$ necessarily contains elements of order 2.

    \item Suppose the order of $g$ is odd. What can you say about the order of $g^2$?

    \item Prove that for all $g, h$ in a group $G$, $|gh| = |hg|$. (Hint: Prove that $|aga^{-1}| = |g|$ for all $a, g$ in $G$.)

    \item In the group of invertible $2 \times 2$ matrices, consider
          \[
              g = \begin{pmatrix} 0 & -1 \\ 1 & 0 \end{pmatrix}, \quad h = \begin{pmatrix} 0 & 1 \\ -1 & 0 \end{pmatrix}.
          \]
          Verify that $|g|=4$, $|h|=3$, and $|gh|=\infty$. [\S1.6]

    \item Give an example showing that $|gh|$ is not necessarily equal to $\text{lcm}(|g|,|h|)$, even if $g$ and $h$ commute. [\S1.6, 1.14]

    \item As a counterpoint to Exercise 1.13, prove that if $g$ and $h$ commute and $\text{gcd}(|g|, |h|) = 1$, then $|gh| = |g||h|$. (Hint: Let $N = |g||h|$; then $g^N = (h^{-1})^N$. What can you say about this element?) [\S1.6, 1.15, \S IV.2.5]

    \item Let $G$ be a commutative group, and let $g \in G$ be an element of maximal finite order, that is, such that if $h \in G$ has finite order, then $|h| \le |g|$. Prove that in fact if $h$ has finite order in $G$, then $|h|$ divides $|g|$. (Hint: Argue by contradiction. If $|h|$ is finite but does not divide $|g|$, then there is a prime integer $p$ such that $|g| = p^m r$, $|h| = p^s s$, with $r$ and $s$ relatively prime to $p$ and $m < n$. Use Exercise 1.14 to compute the order of $g^p h^s$.) [\S2.1, 4.11, IV.6.15]
\end{enumerate}
\section{Examples of groups}
\begin{enumerate}
    \item
\end{enumerate}
\section{The category Grp}
\begin{enumerate}
    \item
\end{enumerate}
\section{Group homomorphisms}
\begin{enumerate}
    \item Check that the function $\pi_m^n$ defined in \S4.1 is well-defined and makes the diagram commute. Verify that it is a group homomorphism. Why is the hypothesis $m \mid n$ necessary? [\S4.1]

    \item Show that the homomorphism $\pi_2^4 \times \pi_2^4$: $C_4 \to C_2 \times C_2$ is not an isomorphism. In fact, is there any isomorphism $C_4 \to C_2 \times C_2$?

    \item Prove that a group of order $n$ is isomorphic to $\mathbb{Z}/n\mathbb{Z}$ if and only if it contains an element of order $n$. [\S4.3]

    \item Prove that no two of the groups $(\mathbb{Z}, +)$, $(\mathbb{Q}, +)$, $(\mathbb{R}, +)$ are isomorphic to one another. Can you decide whether $(\mathbb{R}, +)$, $(\mathbb{C}, +)$ are isomorphic to one another? (Cf. Exercise VI.1.1.)

    \item Prove that the groups $(\mathbb{R} \setminus \{0\}, \cdot)$ and $(\mathbb{C} \setminus \{0\}, \cdot)$ are isomorphic.

    \item We have seen that $(\mathbb{R}, +)$ and $(\mathbb{R}^{>0}, \cdot)$ are isomorphic (Example 4.4). Are the groups $(\mathbb{Q}, +)$ and $(\mathbb{Q}^{>0}, \cdot)$ isomorphic?

    \item Let $G$ be a group. Prove that the function $G \to G$ defined by $g \mapsto g^{-1}$ is a homomorphism if and only if $G$ is abelian. Prove that $g \mapsto g^2$ is a homomorphism if and only if $G$ is abelian.

    \item Let $G$ be a group, and let $g \in G$. Prove that the function $\gamma_g: G \to G$ defined by $(\forall a \in G): \gamma_g(a) = gag^{-1}$ is an automorphism of $G$. (The automorphisms $\gamma_g$ are called `inner' automorphisms of $G$.) Prove that the function $G \to \text{Aut}(G)$ defined by $g \mapsto \gamma_g$ is a homomorphism. Prove that this homomorphism is trivial if and only if $G$ is abelian. [6.7, 7.11, IV.1.5]

    \item Prove that if $m, n$ are positive integers such that $\text{gcd}(m, n)=1$, then $C_{mn} \cong C_m \times C_n$. [\S4.3, 4.10, \S IV.6.1, V.6.8]

    \item Let $p \ne q$ be odd prime integers; show that $(\mathbb{Z}/pq\mathbb{Z})^*$ is not cyclic. (Hint: Use Exercise 4.9 to compute the order $N$ of $(\mathbb{Z}/pq\mathbb{Z})^*$, and show that no element can have order $N$.) [\S4.3]

    \item In due time we will prove the easy fact that if $p$ is a prime integer, then the equation $x^d = 1$ can have at most $d$ solutions in $\mathbb{Z}/p\mathbb{Z}$. Assume this fact, and prove that the multiplicative group $G = (\mathbb{Z}/p\mathbb{Z})^*$ is cyclic. (Hint: Let $g \in G$ be an element of maximal order; use Exercise 1.15 to show that $h^{|g|} = 1$ for all $h \in G$. Therefore...) [\S4.3, 4.15, 4.16, \S IV.6.3]

    \item
          \begin{itemize}
              \item     Compute the order of $[9]_{31}$ in the group $(\mathbb{Z}/31\mathbb{Z})^*$.
              \item Does the equation $x^3 - 9 = 0$ have solutions in $\mathbb{Z}/31\mathbb{Z}$? (Hint: Plugging in all 31 elements of $\mathbb{Z}/31\mathbb{Z}$ is too laborious and will not teach you much. Instead, use the result of the first part: if $c$ is a solution of the equation, what can you say about $|c|$?) [VII.5.15]
          \end{itemize}

    \item Prove that $\text{Aut}_{\mathsf{Grp}}(\mathbb{Z}/2\mathbb{Z} \times \mathbb{Z}/2\mathbb{Z}) \cong S_3$. [IV.5.14]

    \item Prove that the order of the group of automorphisms of a cyclic group $C_n$ is the number of positive integers $r \le n$ that are relatively prime to $n$. (This is called \emph{Euler's $\phi$-function}; cf. Exercise 6.14.) [\S IV.1.4, IV.1.22, \S IV.2.5]

    \item Compute the group of automorphisms of $(\mathbb{Z}, +)$. Prove that if $p$ is prime, then $\text{Aut}_{\mathsf{Grp}}(C_p) \cong C_{p-1}$. (Use Exercise 4.11.) [IV.5.12]

    \item Prove \emph{Wilson's theorem: an integer $p > 1$ is prime if and only if}
          \[ (p-1)! \equiv -1 \pmod p. \]
          (For one direction, use Exercises 1.8 and 4.11. For the other, assume $d$ is a proper divisor of $p$, and note that $d$ divides $(p-1)!$; therefore....) [IV.4.11]

    \item For a few small (but not too small) primes $p$, find a generator of $(\mathbb{Z}/p\mathbb{Z})^*$.

    \item Prove the second part of Proposition 4.8.
\end{enumerate}




\section{Free groups}
\begin{enumerate}
    \item Does the category $\mathscr{F}^A$ defined in \S5.2 have final objects? If so, what are they?

    \item Since trivial groups $T$ are initial in $\mathsf{Grp}$, one may be led to think that $(e, T)$ should be initial in $\mathscr{F}^A$, for every $A$: $e$ would be defined by sending every element of $A$ to the (only) element in $T$; and for any other group $G$, there is a unique homomorphism $T \to G$. Explain why $(e, T)$ is not initial in $\mathscr{F}^A$ (unless $A = \emptyset$).

    \item Use the universal property of free groups to prove that the map $j: A \to F(A)$ is injective, for all sets $A$. (Hint: It suffices to show that for every two elements $a, b$ of $A$ there is a group $G$ and a set-function $f: A \to G$ such that $f(a) \ne f(b)$. Why? How do you construct $f$ and $G$?) [III.6.3]

    \item In the `concrete' construction of free groups, one can try to reduce words by performing cancellations in any order; the process of `elementary reductions' used in the text (that is, from left to right) is only one possibility. Prove that the result of iterating cancellations on a word is independent of the order in which the cancellations are performed. Deduce the associativity of the product in $F(A)$ from this. [\S5.3]

    \item Verify explicitly that $H^{\oplus A}$ is a group.

    \item Prove that the group $F(\{x, y\})$ (visualized in Example 5.3) is a coproduct $\mathbb{Z} \ast \mathbb{Z}$ of $\mathbb{Z}$ by itself in the category $\mathsf{Grp}$. (Hint: With due care, the universal property for one turns into the universal property for the other.) [\S3.4, 3.7, 5.7]

    \item Extend the result of Exercise 5.6 to free groups $F(\{x_1, \dots, x_n\})$ and to free abelian groups $F^{\text{ab}}(\{x_1, \dots, x_n\})$. [\S5.4]

    \item Still more generally, prove that $F(A \coprod B) = F(A) \ast F(B)$ and that $F^{\text{ab}}(A \coprod B) = F^{\text{ab}}(A) \oplus F^{\text{ab}}(B)$ for all sets $A, B$. (That is, the constructions $F, F^{\text{ab}}$ `preserve coproducts'.)

    \item Let $G = \mathbb{Z}^{\oplus \mathbb{N}}$. Prove that $G \times G \cong G$.

    \item Let $F = F^{\text{ab}}(A)$.
          \begin{itemize}
              \item Define an equivalence relation $\sim$ on $F$ by setting $f' \sim f$ if and only if $f - f' = 2g$ for some $g \in F$. Prove that $F/\sim$ is a finite set if and only if $A$ is finite, and in that case $|F/\sim| = 2^{|A|}$.
              \item Assume $F^{\text{ab}}(B) \cong F^{\text{ab}}(A)$. If $A$ is finite, prove that $B$ is also, and that $A \cong B$ as sets. (This result holds for free groups as well, and without any finiteness hypothesis. See Exercises 7.13 and VI.1.20.)
          \end{itemize}


          [7.4, 7.13]
\end{enumerate}
\section{Subgroups}
\begin{enumerate}
    \item (If you know about matrices.) The group of invertible $n \times n$ matrices with entries in $\mathbb{R}$ is denoted $\text{GL}_n(\mathbb{R})$ (Example I.5). Similarly, $\text{GL}_n(\mathbb{C})$ denotes the group of $n \times n$ invertible matrices with complex entries. Consider the following sets of matrices:
          \begin{itemize}
              \item $\text{SL}_n(\mathbb{R}) = \{ M \in \text{GL}_n(\mathbb{R}) \mid \det(M) = 1 \}$;
              \item $\text{SL}_n(\mathbb{C}) = \{ M \in \text{GL}_n(\mathbb{C}) \mid \det(M) = 1 \}$;
              \item $\text{O}_n(\mathbb{R}) = \{ M \in \text{GL}_n(\mathbb{R}) \mid M M^t = M^t M = I_n \}$;
              \item $\text{SO}_n(\mathbb{R}) = \{ M \in \text{O}_n(\mathbb{R}) \mid \det(M) = 1 \}$;
              \item $\text{U}(n) = \{ M \in \text{GL}_n(\mathbb{C}) \mid M M^\dagger = M^\dagger M = I_n \}$;
              \item $\text{SU}(n) = \{ M \in \text{U}(n) \mid \det(M) = 1 \}$.
          \end{itemize}
          Here $I_n$ stands for the $n \times n$ identity matrix, $M^t$ is the transpose of $M$, $M^\dagger$ is the conjugate transpose of $M$, and $\det(M)$ denotes the determinant\footnote{If you are not familiar with some of these notions, that's ok: leave this exercise and similar ones alone if that is the case. We will come back to linear algebra and matrices in Chapter VI and following.} of $M$. Find all possible inclusions among these sets, and prove that in every case the smaller set is a subgroup of the larger one.

          These sets of matrices have compelling geometric interpretations: for example, $\text{SO}_3(\mathbb{R})$ is the group of `rotations' in $\mathbb{R}^3$. [8.8, 9.1, III.1.4, VI.6.16]

    \item Prove that the set of $2 \times 2$ matrices
          \[ \begin{pmatrix} a & b \\ 0 & d \end{pmatrix} \]
          with $a, b, d \in \mathbb{C}$ and $ad \ne 0$ is a subgroup of $\text{GL}_2(\mathbb{C})$. More generally, prove that the set of $n \times n$ complex matrices $(a_{ij})_{1 \le i,j \le n}$ with $a_{ij}=0$ for $i > j$ and $a_{11} \cdots a_{nn} \ne 0$ is a subgroup of $\text{GL}_n(\mathbb{C})$. (These matrices are called `upper triangular', for evident reasons.) [IV.1.20]

    \item Prove that every matrix in $\text{SU}(2)$ may be written in the form
          \[ \begin{pmatrix} a+bi & c+di \\ -c+di & a-bi \end{pmatrix} \]
          where $a, b, c, d \in \mathbb{R}$ and $a^2+b^2+c^2+d^2=1$. (Thus, $\text{SU}(2)$ may be realized as a three-dimensional sphere embedded in $\mathbb{R}^4$; in particular, it is simply connected.) [8.9, III.2.5]

    \item Let $G$ be a group, and let $g \in G$. Verify that the image of the exponential map $\epsilon_g: \mathbb{Z} \to G$ is a cyclic group (in the sense of Definition 4.7). [\S6.3, \S7.5]

    \item Let $G$ be a commutative group, and let $n > 0$ be an integer. Prove that $\{g^n \mid g \in G\}$ is a subgroup of $G$. Prove that this is not necessarily the case if $G$ is not commutative.

    \item Prove that the union of a family of subgroups of a group $G$ is not necessarily a subgroup of $G$. In fact:
          \begin{itemize}
              \item Let $H, H'$ be subgroups of a group $G$. Prove that $H \cup H'$ is a subgroup of $G$ only if $H \subseteq H'$ or $H' \subseteq H$.
              \item On the other hand, let $H_0 \subseteq H_1 \subseteq H_2 \subseteq \cdots$ be subgroups of a group $G$. Prove that $\bigcup_{i \ge 0} H_i$ is a subgroup of $G$.
          \end{itemize}

    \item Show that inner automorphisms (cf. Exercise 4.8) form a subgroup of $\text{Aut}(G)$. This subgroup is denoted $\text{Inn}(G)$. Prove that $\text{Inn}(G)$ is cyclic if and only if $\text{Inn}(G)$ is trivial if and only if $G$ is abelian. (Hint: Assume that $\text{Inn}(G)$ is cyclic; with notation as in Exercise 4.8, this means that there exists an element $a \in G$ such that $\forall g \in G \exists k \in \mathbb{Z}: \gamma_g = \gamma_a^k$. In particular, $gag^{-1} = a^k a^{-k}$. Thus $a$ commutes with every $g$ in $G$. Therefore....) Deduce that if $\text{Aut}(G)$ is cyclic, then $G$ is abelian. [7.10, IV.1.5]

    \item Prove that an abelian group $G$ is finitely generated if and only if there is a surjective homomorphism
      \[
      \underbrace{\mathbb{Z} \oplus \cdots \oplus \mathbb{Z}}_{\text{$n$ times}}
      \twoheadrightarrow G
      \]
      for some $n$.

    \item Prove that every finitely generated subgroup of $\mathbb{Q}$ is cyclic. Prove that $\mathbb{Q}$ is not finitely generated.

    \item The set of $2 \times 2$ matrices with integer entries and determinant 1 is denoted $\text{SL}_2(\mathbb{Z})$:
          \[ \text{SL}_2(\mathbb{Z}) = \left\{ \begin{pmatrix} a & b \\ c & d \end{pmatrix} \mid a, b, c, d \in \mathbb{Z}, ad-bc=1 \right\}. \]
          Prove that $\text{SL}_2(\mathbb{Z})$ is generated by the matrices
          \[ s = \begin{pmatrix} 0 & -1 \\ 1 & 0 \end{pmatrix} \quad \text{and} \quad t = \begin{pmatrix} 1 & 1 \\ 0 & 1 \end{pmatrix}. \]
          (Hint: This is a little tricky. Let $H$ be the subgroup generated by $s$ and $t$. Given a matrix $m = \begin{pmatrix} a & b \\ c & d \end{pmatrix}$ in $\text{SL}_2(\mathbb{Z})$, it suffices to show that you can obtain the identity by multiplying $m$ by suitably chosen elements of $H$. Prove that $\begin{pmatrix} 1 & -q \\ 0 & 1 \end{pmatrix}$ and $\begin{pmatrix} 1 & 0 \\ -q & 1 \end{pmatrix}$ are in $H$, and note that
          \[ \begin{pmatrix} a & b \\ c & d \end{pmatrix} \begin{pmatrix} 1 & -q \\ 0 & 1 \end{pmatrix} = \begin{pmatrix} a & b-qa \\ c & d-qc \end{pmatrix} \quad \text{and} \quad \begin{pmatrix} a & b \\ c & d \end{pmatrix} \begin{pmatrix} 1 & 0 \\ -q & 1 \end{pmatrix} = \begin{pmatrix} a-bq & b \\ c-dq & d \end{pmatrix}. \]
          Note that if $c$ and $d$ are both nonzero, one of these two operations may be used to decrease the absolute value of one of them. Argue that suitable applications of these operations reduce to the case in which $c=0$ or $d=0$. Prove directly that $m \in H$ in that case.) [7.5]

    \item Since direct sums are coproducts in $\mathsf{Ab}$, the classification theorem for abelian groups mentioned in the text says that every finitely generated abelian group is a coproduct of cyclic groups in $\mathsf{Ab}$. The reader is may be tempted to conjecture that every finitely generated group is a coproduct of cyclic groups in $\mathsf{Grp}$. Show that this is not the case, by proving that $S_3$ is not a coproduct of cyclic groups.

    \item Let $m, n$ be positive integers, and consider the subgroup $\langle m, n \rangle$ of $\mathbb{Z}$ they generate. By Proposition 6.9,
          \[ \langle m, n \rangle = d\mathbb{Z} \]
          for some positive integer $d$. What is $d$ in relation to $m, n$?

    \item Draw and compare the lattices of subgroups of $C_2 \times C_2$ and $C_4$. Draw the lattice of subgroups of $S_3$, and compare it with the one for $C_6$. [7.1]

    \item If $m$ is a positive integer, denote by $\phi(m)$ the number of positive integers $r \le m$ that are relatively prime to $m$ (that is, for which the $\text{gcd}$ of $r$ and $m$ is 1); this is called Euler's $\phi$- (or `totient') function. For example, $\phi(12)=4$. In other words, $\phi(m)$ is the order of the group $(\mathbb{Z}/m\mathbb{Z})^*$. cf. Proposition 2.6.
          Put together the following observations:
          \begin{itemize}
              \item $\phi(m) = $ the number of generators of $C_m$,
              \item every element of $C_n$ generates a subgroup of $C_n$,
              \item the discussion following Proposition 6.11 (in particular, every subgroup of $C_n$ is isomorphic to $C_m$, for some $m \mid n$),
          \end{itemize}
          to obtain a proof of the formula
          \[ \sum_{m>0, m \mid n} \phi(m) = n. \]
          (For example, $\phi(1)+\phi(2)+\phi(3)+\phi(4)+\phi(6)+\phi(12) = 1+1+2+2+2+4 = 12$.) [4.14, \S6.4, 8.15, V.6.8, \S VII.5.2]

    \item Prove that if a group homomorphism $\varphi: G \to G'$ has a left-inverse, that is, a group homomorphism $\psi: G' \to G$ such that $\psi \circ \varphi = \text{id}_G$, then $\varphi$ is a monomorphism. [\S6.5, 6.16]

    \item Counterpoint to Exercise 6.15: the homomorphism $\varphi: \mathbb{Z}/3\mathbb{Z} \to S_3$ given by
          \[ \varphi([0]) = \begin{pmatrix} 1 & 2 & 3 \\ 1 & 2 & 3 \end{pmatrix}, \quad \varphi([1]) = \begin{pmatrix} 1 & 2 & 3 \\ 3 & 1 & 2 \end{pmatrix}, \quad \varphi([2]) = \begin{pmatrix} 1 & 2 & 3 \\ 2 & 3 & 1 \end{pmatrix} \]
          is a monomorphism; show that it has no left-inverse in $\mathsf{Grp}$. (Knowing about normal subgroups will make this problem particularly easy.) [\S6.5]
\end{enumerate}
\section{Quotient groups}
\begin{enumerate}
    \item
\end{enumerate}
\section{Canonical decomposition and Lagrange's theorem}
\begin{enumerate}
    \item If a group $H$ may be realized as a subgroup of two groups $G_1$ and $G_2$ and if
          \[ \frac{G_1}{H} \cong \frac{G_2}{H}, \]
          does it follow that $G_1 \cong G_2$? Give a proof or a counterexample.

    \item Extend Example 8.6 as follows. Suppose $G$ is a group and $H \subseteq G$ is a subgroup of index 2, that is, such that there are precisely two (say, left-) cosets of $H$ in $G$. Prove that $H$ is normal in $G$. [9.11, IV.1.16]

    \item Prove that every finite group is finitely presented.

    \item Prove that $(a, b^2, ab)$ is a presentation of the dihedral group $D_{2n}$. (Hint: With respect to the generators defined in Exercise 2.5, set $a=x$ and $b=y$; prove you can get the relations given here from the ones you obtained in Exercise 2.5, and conversely.)

    \item Let $a, b$ be distinct elements of order 2 in a group $G$, and assume that $ab$ has finite order $n \ge 2$. Prove that the subgroup generated by $a$ and $b$ in $G$ is isomorphic to the dihedral group $D_{2n}$. (Use the previous exercise.)

    \item Let $G$ be a group, and let $A$ be a set of generators for $G$; assume $A$ is finite. The corresponding Cayley graph is a directed graph whose set of vertices is in one-to-one correspondence with $G$, and two vertices $g_1, g_2$ are connected by an edge if $g_2 = g_1 a$ for an $a \in A$; this edge may be labeled $a$ and oriented from $g_1$ to $g_2$. For example, the graph drawn in Example 3.3 for the free group $F(\{x,y\})$ on two generators $x, y$ is the corresponding Cayley graph (with the convention that horizontal edges are labeled $x$ and point to the right and vertical edges are labeled $y$ and point up).

          Prove that if a Cayley graph of a group is a tree, then the group is free. Conversely, prove that free groups admit Cayley graphs that are trees. [\S5.3, 9.15]

    \item Let $(A, \Re)$, resp. $(A', \Re')$, be a presentation for a group $G$, resp., $G'$ (cf. \S8.2); we may assume that $A, A'$ are disjoint. Prove that the group $G \ast G'$ presented by
          \[ (A \cup A' \mid \Re \cup \Re') \]
          satisfies the universal property for the coproduct of $G$ and $G'$ in $\mathsf{Grp}$. (Use the universal properties of both free groups and quotients to construct natural homomorphisms $G \to G \ast G'$, $G' \to G \ast G'$.) [\S3.4, \S8.2, 9.14]

    \item (If you know about matrices (cf. Exercise 6.1)). Prove that $\text{SL}_n(\mathbb{R})$ is a normal subgroup of $\text{GL}_n(\mathbb{R})$, and `compute' $\text{GL}_n(\mathbb{R})/\text{SL}_n(\mathbb{R})$ as a well-known group. [VI.3.3]

    \item (Ditto.) Prove that $\text{SO}_3(\mathbb{R}) \cong \text{SU}(2)/\{\pm I_2\}$, where $I_2$ is the identity matrix. (Hint: It so happens that every matrix in $\text{SO}_3(\mathbb{R})$ can be written in the form
          \[
              \begin{pmatrix}
                  a^2+b^2-c^2-d^2 & 2(bc-ad)        & 2(bd+ac)        \\
                  2(bc+ad)        & a^2-b^2+c^2-d^2 & 2(cd-ab)        \\
                  2(bd-ac)        & 2(cd+ab)        & a^2-b^2-c^2+d^2
              \end{pmatrix}
          \]
          where $a, b, c, d \in \mathbb{R}$ and $a^2+b^2+c^2+d^2=1$. Proving this fact is not hard, but at this stage you will probably find it computationally demanding. Feel free to assume this, and use Exercise 6.3 to construct a surjective homomorphism $\text{SU}(2) \to \text{SO}_3(\mathbb{R})$; compute the kernel of this homomorphism.)

          If you know a little topology, you can now conclude that the fundamental group of $\text{SO}_3(\mathbb{R})$ is $C_2$. [9.1, VI.1.3]

    \item View $\mathbb{Z} \times \mathbb{Z}$ as a subgroup of $\mathbb{R} \times \mathbb{R}$:
          \[
              \begin{tikzpicture}
                  \draw[->] (-2.5,0) -- (2.5,0);
                  \draw[->] (0,-2.5) -- (0,2.5);
                  \foreach \x in {-2,-1,0,1,2}
                  \foreach \y in {-2,-1,0,1,2}
                  \node[fill,circle,inner sep=1.5pt] at (\x,\y) {};
              \end{tikzpicture}
          \]
          Describe the quotient
          \[ \frac{\mathbb{R} \times \mathbb{R}}{\mathbb{Z} \times \mathbb{Z}} \]
          in terms analogous to those used in Example 8.7. (Can you `draw a picture' of this group? Cf. Exercise 1.1.6.)

    \item (Notation as in Proposition 8.10.) Prove `by hand' (that is, without invoking universal properties) that $N$ is normal in $G$ if and only if $N/H$ is normal in $G/H$.

    \item (Notation as in Proposition 8.11.) Prove `by hand' (that is, by using Proposition 6.2) that $HK$ is a subgroup of $G$ if $H$ is normal.

    \item Let $G$ be a finite group, and assume $|G|$ is odd. Prove that every element of $G$ is a square. [8.14]

    \item Generalize the result of Exercise 8.13: if $G$ is a group of order $n$ and $k$ is an integer relatively prime to $n$, then the function $G \to G, g \mapsto g^k$ is surjective.

    \item Let $a, n$ be positive integers, with $a>1$. Prove that $n$ divides $\phi(a^n-1)$, where $\phi$ is Euler's $\phi$-function; see Exercise 6.14. (Hint: Example 8.15.)

    \item Prove that in every category $\mathsf{C}$ the products $A \times B$ and $B \times A$ are isomorphic, if they exist. (Hint: Observe that they both satisfy the universal property for the product of $A$ and $B$; then use Proposition 5.4.)

    \item Assume $G$ is a finite abelian group, and let $p$ be a prime divisor of $|G|$. Prove that there exists an element in $G$ of order $p$. (Hint: Let $g \neq e$ be an element of $G$, and consider the subgroup $\langle g \rangle$; use the fact that this subgroup is cyclic to show that there is an element $h \in \langle g \rangle$ of prime order $q$. If $q=p$, you are done; otherwise, use the quotient $G/\langle h \rangle$ and induction.) [\S8.5, 8.18, 8.20, \S IV.2.1]

    \item Let $G$ be an abelian group of order $2n$, where $n$ is odd. Prove that $G$ has exactly one element of order 2. (It has at least one, for example by Exercise 8.17. Use Lagrange's theorem to establish that it cannot have more than one.) Does the same conclusion hold if $G$ is not necessarily commutative?

    \item Let $G$ be a finite group, and let $d$ be a proper divisor of $|G|$. Is it necessarily true that there exists an element of $G$ of order $d$? Give a proof or a counterexample.

    \item Assume $G$ is a finite abelian group, and let $d$ be a divisor of $|G|$. Prove that there exists a subgroup $H \subseteq G$ of order $d$. (Hint: induction; use Exercise 8.17.) [\S IV.2.2]

    \item Let $H, K$ be subgroups of a group $G$. Construct a bijection between the set of cosets $hK$ with $h \in H$ and the set of left-cosets of $H \cap K$ in $H$. If $H$ and $K$ are finite, prove that
          \[ |HK| = \frac{|H| \cdot |K|}{|H \cap K|}. \]
          [\S8.5, \S IV.4.4]

    \item Let $\varphi: G \to G'$ be a group homomorphism, and let $N$ be the smallest normal subgroup containing $\text{im}\,\varphi$. Prove that $G'/N$ satisfies the universal property of coker $\varphi$ in $\mathsf{Grp}$. [\S8.6]

    \item Consider the subgroup
          \[ H = \left\{ \begin{pmatrix} 1 & 2 & 3 \\ 1 & 2 & 3 \end{pmatrix}, \begin{pmatrix} 1 & 2 & 3 \\ 2 & 1 & 3 \end{pmatrix} \right\} \]
          of $S_3$. Show that the cokernel of the inclusion $H \hookrightarrow S_3$ is trivial, although $H \hookrightarrow S_3$ is not surjective. [\S8.6]

    \item Show that epimorphisms in $\mathsf{Grp}$ do not necessarily have right-inverses. [\S I.4.2]

    \item Let $H$ be a commutative normal subgroup of $G$. Construct an interesting homomorphism from $G/H$ to $\text{Aut}(H)$. (Cf. Exercise 7.10.)
\end{enumerate}
\section{Group actions}
\begin{enumerate}
    \item (Once more, if you are already familiar with a little linear algebra...) The matrix groups listed in Exercise 6.1 all come with evident actions on a vector space: if $M$ is an $n \times n$ matrix with (say) real entries, multiplication to the right by a column $n$-vector $v$ returns a column $n$-vector $Mv$, and this defines a left-action on $\mathbb{R}^n$ viewed as the space of column $n$-vectors.
          \begin{itemize}
              \item Prove that, through this action, matrices $M \in \text{O}_n(\mathbb{R})$ preserve lengths and angles in $\mathbb{R}^n$.
              \item Find an interesting action of $\text{SU}(2)$ on $\mathbb{R}^3$. (Hint: Exercise 8.9.)
          \end{itemize}

    \item The effect of the matrices
          \[ \begin{pmatrix} 1 & 0 \\ 0 & -1 \end{pmatrix}, \begin{pmatrix} 0 & 1 \\ -1 & 0 \end{pmatrix} \]
          on the plane is to respectively flip the plane about the $x$-axis and to rotate it $90^\circ$ clockwise about the origin. With this in mind, construct an action of $D_8$ on $\mathbb{R}^2$.

    \item If $G = (G, \cdot)$ is a group, we can define an `opposite' group $G^\mathsf{o} = (G, \bullet)$ supported on the same set $G$, by prescribing
          \[ (\forall g, h \in G): g \bullet h = h \cdot g. \]
          \begin{itemize}
              \item Verify that $G^\mathsf{o}$ is indeed a group.
              \item Show that the `identity': $G^\mathsf{o} \to G, g \mapsto g$ is an isomorphism if and only if $G$ is commutative.
              \item Show that $G^\mathsf{o} \cong G$ (even if $G$ is not commutative!).
              \item Show that giving a right-action of $G$ on a set $A$ is the same as giving a homomorphism $G^\mathsf{o} \to S_A$ (with the convention for $S_A$ adopted in this section, see the beginning of \S9.2), that is, a left-action of $G^\mathsf{o}$ on $A$.
              \item Show that the notions of left- and right-actions coincide `on the nose' for commutative groups. (That is, if $(g, a) \mapsto ga$ defines a right-action of a commutative group $G$ on a set $A$, then setting $ga = ag$ defines a left-action).
              \item For any group $G$, explain how to turn a right-action of $G$ into a left-action of $G$. (Note that the simple `flip' $ga = ag$ does not work in general if $G$ is not commutative.)
          \end{itemize}

    \item As mentioned in the text, right-multiplication defines a right-action of a group on itself. Find another natural right-action of a group on itself.

    \item Prove that the action by left-multiplication of a group on itself is free.

    \item Let $O$ be an orbit of an action of a group $G$ on a set. Prove that the induced action of $G$ on $O$ is transitive.

    \item Prove that stabilizers are indeed subgroups.

    \item For $G$ a group, verify that $G$-$\mathsf{Set}$ is indeed a category, and verify that the isomorphisms in $G$-$\mathsf{Set}$ are precisely the equivariant bijections.

    \item Prove that $G$-$\mathsf{Set}$ has products and coproducts and that every object of $G$-$\mathsf{Set}$ is a coproduct of objects of the type $G/H = \{\text{left-cosets of } H\}$, where $H$ is a subgroup of $G$ and $G$ acts on $G/H$ by left-multiplication.

    \item Let $H$ be any subgroup of a group $G$. Prove that there is a bijection between the set $G/H$ of left-cosets of $H$ and the set $H \backslash G$ of right-cosets of $H$ in $G$. (Hint: $G$ acts on the right on the set of right-cosets; use Exercise 9.3 and Proposition 9.9.)

    \item Let $G$ be a finite group, and let $H$ be a subgroup of index $p$, where $p$ is the smallest prime dividing $|G|$. Prove that $H$ is normal in $G$, as follows:
          \begin{itemize}
              \item Interpret the action of $G$ on $G/H$ by left-multiplication as a homomorphism $\sigma: G \to S_p$.
              \item Then $\text{Ker}\,\sigma$ is (isomorphic to) a subgroup of $S_p$. What does this say about the index of $\text{Ker}\,\sigma$ in $G$?
              \item Show that $\text{Ker}\,\sigma \subseteq H$.
              \item Conclude that $H = \text{Ker}\,\sigma$, by index considerations.
          \end{itemize}
          Thus $H$ is a kernel, proving that it is normal. (This exercise generalizes the result of Exercise 8.2.) [9.12]

    \item Generalize the result of Exercise 9.11, as follows. Let $G$ be a group, and let $H \subseteq G$ be a subgroup of index $n$. Prove that $H$ contains a subgroup $K$ that is normal in $G$ and such that $[G: K]$ divides the gcd of $|G|$ and $n!$. (In particular, $[G: K] \le n!$.) [IV.2.23]

    \item Prove `by hand' that for all subgroups $H$ of a group $G$ and $\forall g \in G$, $G/H$ and $G/(gHg^{-1})$ (endowed with the action of $G$ by left-multiplication) are isomorphic in $G$-$\mathsf{Set}$. [\S9.3]

    \item Prove that the modular group $\text{PSL}_2(\mathbb{Z})$ is isomorphic to the coproduct $C_2 \ast C_3$. (Recall that the modular group $\text{PSL}_2(\mathbb{Z})$ is generated by $x = \begin{pmatrix} 0 & -1 \\ 1 & 0 \end{pmatrix}$ and $y = \begin{pmatrix} 1 & 1 \\ 0 & 1 \end{pmatrix}$. Exercise 7.5). The task is to prove that $x$ and $y$ satisfy no other relation: this will show that $\text{PSL}_2(\mathbb{Z})$ is presented by $(x, y \mid x^2, y^3)$, and we have agreed that this is a presentation for $C_2 \ast C_3$ (Exercise 3.8 or 8.7). Reduce this to verifying that no products $(y^{\pm 1}x)(y^{\pm 1}x) \cdots (y^{\pm 1}x)$ or $(y^{\pm 1}x)(y^{\pm 1}x) \cdots (y^{\pm 1}x) y^{\pm 1}$ with one or more factors can equal the identity. This latter verification is traditionally carried out by cleverly exploiting an action\footnote{The modular group acts on $\mathbb{C} \cup \{\infty\}$ by Möbius transformations. The observation that it suffices to act on $\mathbb{R} \setminus \mathbb{Q}$ for the purpose of this verification is due to Roger Alperin.} on the set of irrational real numbers by
          \[ \begin{pmatrix} a & b \\ c & d \end{pmatrix} r = \frac{ar+b}{cr+d} \]
          Check that this does define an action of $\text{PSL}_2(\mathbb{Z})$, and note that
          \[ y(r) = 1-\frac{1}{r}, \quad y^{-1}(r) = 1-\frac{1}{r'}, \quad yx(r) = 1+\frac{1}{r'}, \quad y^{-1}x(r) = \frac{1}{r} \]

    \item Prove that every (finitely generated) group $G$ acts freely on any corresponding Cayley graph. (Cf. Exercise 8.6. Actions on a directed graph are defined as actions on the set of vertices preserving incidence: if the vertices $v_1, v_2$ are connected by an edge, then so must be $gv_1, gv_2$ for every $g \in G$.) In particular, conclude that every free group acts freely on a tree. [9.16]

    \item The converse of the last statement in Exercise 9.15 is also true: only free groups can act freely on a tree. Assuming this, prove that every subgroup of a free group (on a finite set) is free. [\S6.4]

    \item Consider $G$ as a $G$-set, by acting with left-multiplication. Prove that $\text{Aut}_{G\text{-}\mathsf{Set}}(G) \cong G$. [\S2.1]

    \item Show how to construct a groupoid carrying the information of the action of a group $G$ on a set $A$. (Hint: $A$ will be the set of objects of the groupoid. What will be the morphisms?)
\end{enumerate}
\section{Group objects in categories}
\begin{enumerate}
    \item Define all the unnamed maps appearing in the diagrams in the definition of group object, and prove they are indeed isomorphisms when so indicated. (For the projection $1 \times G \to G$, what is left to prove is that the composition
          \[ 1 \times G \to G \to 1 \times G \]
          is the identity, as mentioned in the text.)

    \item Show that \emph{groups}, as defined in \S1.2, are `group objects in the category of sets'. [\S10.1]

    \item Let $(G, \cdot)$ be a group, and suppose $\circ: G \times G \to G$ is a group homomorphism (w.r.t. $\cdot$) such that $(G, \circ)$ is \emph{also} a group. Prove that $\circ$ and $\cdot$ coincide. (Hint: First prove that the identity with respect to the two operations must be the same.)

    \item Prove that every \emph{abelian} group has exactly one structure of group object \emph{in the category} $\mathsf{Ab}$.

    \item By the previous exercise, a group object in $\mathsf{Ab}$ is nothing other than an abelian group. What is a group object in $\mathsf{Grp}$?
\end{enumerate}

% Chapter 3
\chapter{Rings and modules}
\section{Definition of ring}
\begin{enumerate}
    \item
\end{enumerate}
\section{The category Ring}
\begin{enumerate}
    \item
\end{enumerate}
\section{Ideals and quotient rings}
\begin{enumerate}
    \item
\end{enumerate}


\section{Ideals and quotients: Remarks and examples. Prime and maximal ideals}
\begin{enumerate}
    \item
\end{enumerate}

\section{Modules over a ring}
\begin{enumerate}
    \item
\end{enumerate}

\section{Products, coproducts, etc., in R-Mod}
\begin{enumerate}
    \item
\end{enumerate}

\section{Complexes and homology}
\begin{enumerate}
    \item
\end{enumerate}


% Chapter 4
\chapter{Groups, second encounter}
\section{The conjugation action}
\begin{enumerate}
    \item
\end{enumerate}
\section{The Sylow theorems}
\begin{enumerate}
    \item
\end{enumerate}

\section{Composition series and solvability}
\begin{enumerate}
    \item
\end{enumerate}

\section{The symmetric group}
\begin{enumerate}
    \item
\end{enumerate}

\section{Products of groups}
\begin{enumerate}
    \item
\end{enumerate}

\section{Finite abelian groups}
\begin{enumerate}
    \item
\end{enumerate}

% Add more chapters as needed
% \chapter{Chapter 3: Calculus I}
% \section{Definition of ring}
\begin{enumerate}
    \item
\end{enumerate}
% \section{The category Ring}
\begin{enumerate}
    \item
\end{enumerate}

% --- End Document ---
\end{document}