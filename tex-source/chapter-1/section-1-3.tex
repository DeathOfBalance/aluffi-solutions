\section{Categories}
\begin{enumerate}
    \item Let $\mathsf{C}$ be a category. Consider a structure $\mathsf{C}^{\mathrm{op}}$ with
          \begin{itemize}
              \item $\mathrm{Obj}(\mathsf{C}^{\mathrm{op}}) := \mathrm{Obj}(\mathsf{C})$;
              \item for $A, B$ objects of $\mathsf{C}^{\mathrm{op}}$ (hence objects of $\mathsf{C}$), $\mathrm{Hom}_{\mathsf{C}^{\mathrm{op}}}(A, B) := \mathrm{Hom}_{\mathsf{C}}(B, A)$.
          \end{itemize}
          Show how to make this into a category (that is, define composition of morphisms in $\mathsf{C}^{\mathrm{op}}$ and verify the properties listed in \S3.1).

          Intuitively, the `opposite' category $\mathsf{C}^{\mathrm{op}}$ is simply obtained by `reversing all the arrows' in $\mathsf{C}$. [\S5.1, \S VIII.1.1, \S IX.1.2, IX.1.10]

    \item If $A$ is a finite set, how large is $\mathrm{End}_{\mathsf{Set}}(A)$?
        \begin{solution}
            By Example 3.2 and Exercise I.2.10, we conclude that $|\mathrm{End}_{\mathsf{Set}}(A)|=n^n$.
        \end{solution}
    \item Formulate precisely what it means to say that $1_A$ is an identity with respect to composition in Example 3.3, and prove this assertion. [\S3.2]

    \item Can we define a category in the style of Example 3.3 using the relation $<$ on the set $\mathbb{Z}$?
    
    \begin{solution}
        No, because the relation $<$ is not reflexive. This implies that the identity morphism $1_a$ cannot be defined for any object $a$ in this category, since there is no element $a \in \mathbb{Z}$ such that $a < a$. In a category, every object must have an identity morphism, which is not the case here.
    \end{solution}

    \item Explain in what sense Example 3.4 is an instance of the categories considered in Example 3.3. [\S3.2]

    \item (Assuming some familiarity with linear algebra.) Define a category $\mathsf{V}$ by taking $\mathrm{Obj}(\mathsf{V}) = \mathbb{N}$ and letting $\mathrm{Hom}_{\mathsf{V}}(n, m) = \text{the set of } m \times n \text{ matrices with real entries}$, for all $n, m \in \mathbb{N}$. (I will leave the reader the task of making sense of a matrix with 0 rows or columns.) Use product of matrices to define composition. Does this category `feel' familiar? [\S VI.2.1, \S VIII.1.3]

    \item Define carefully objects and morphisms in Example 3.7, and draw the diagram corresponding to composition. [\S3.2]

    \item A \textit{subcategory} $\mathsf{C}'$ of a category $\mathsf{C}$ consists of a collection of objects of $\mathsf{C}$ with sets of morphisms $\mathrm{Hom}_{\mathsf{C}'}(A, B) \subseteq \mathrm{Hom}_{\mathsf{C}}(A, B)$ for all objects $A, B$ in $\mathrm{Obj}(\mathsf{C}')$, such that identities and compositions in $\mathsf{C}$ make $\mathsf{C}'$ into a category. A subcategory $\mathsf{C}'$ is full if $\mathrm{Hom}_{\mathsf{C}'}(A, B) = \mathrm{Hom}_{\mathsf{C}}(A, B)$ for all $A, B$ in $\mathrm{Obj}(\mathsf{C}')$. Construct a category of \textit{infinite sets} and explain how it may be viewed as a full subcategory of $\mathsf{Set}$. [4.4, \S VI.1.1, \S VIII.1.3]

    \item An alternative to the notion of \textit{multiset} introduced in \S2.2 is obtained by considering sets endowed with equivalence relations; equivalent elements are taken to be multiple instances of elements `of the same kind'. Define a notion of morphism between such enhanced sets, obtaining a category $\mathsf{MSet}$ containing (a `copy' of) $\mathsf{Set}$ as a full subcategory. (There may be more than one reasonable way to do this! This is intentionally an open-ended exercise.) Which objects in $\mathsf{MSet}$ determine ordinary multisets as defined in \S2.2 and how? Spell out what a morphism of multisets would be from this point of view. (There are several natural notions of morphisms of multisets. Try to define morphisms in $\mathsf{MSet}$ so that the notion you obtain for ordinary multisets captures your intuitive understanding of these objects.) [\S2.2, \S3.2, 4.5]

    \item Since the objects of a category $\mathsf{C}$ are not (necessarily interpreted as) sets, it is not clear how to make sense of a notion of `subobject' in general, extrapolating the notion of \emph{subset}. In some situations it \textit{does} make sense to talk about subobjects, and the subobjects of any given object $A$ in $\mathsf{C}$ are in one-to-one correspondence with the morphisms $A \to \Omega$ for a fixed, special object $\Omega$ of $\mathsf{C}$, called a \textit{subobject classifier}. Show that $\mathsf{Set}$ has a subobject classifier.
    
    \begin{solution}
        The subobject classifier in $\mathsf{Set}$ is $\Omega = \{0,1\}$ (up to isomorphism). Indeed, there is a bijection between subsets (subobjects in the category of sets) $B \subseteq A$ and functions $A \to \Omega$, namely, $B$ corresponds to its indicator function $\chi_B:A \to \Omega$ defined by $\chi_B(a) = 1$ if $a \in B$, and $0$ otherwise. This function is a morphism in $\mathsf{Set}$, and every morphism from $A$ to $\Omega$ arises in this way from a unique subset of $A$. Thus, $\Omega$ serves as a subobject classifier in $\mathsf{Set}$.
    \end{solution}

    \item Draw the relevant diagrams and define composition and identities for the category $\mathsf{C}^{A,B}$ mentioned in Example 3.9. Do the same for the category $\mathsf{C}^{\alpha,\beta}$ mentioned in Example 3.10. [\S5.5, 5.12]
\end{enumerate}