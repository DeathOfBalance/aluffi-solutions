\section{Universal properties}
\begin{enumerate}
    \item Prove that a final object in a category $\mathsf{C}$ is initial in the opposite category $\mathsf{C}^{\text{op}}$ (cf. Exercise 3.1).
    \item Prove that $\emptyset$ is the unique initial object in $\mathsf{Set}$. [\S5.1]
    \item Prove that final objects are unique up to isomorphism. [\S5.1]
    \item What are initial and final objects in the category of `pointed sets' (Example 3.8)? Are they unique?
    \item What are the final objects in the category considered in \S5.3? [\S5.3]
    \item Consider the category corresponding to endowing (as in Example 3.3) the set $\mathbb{Z}^+$ of positive integers with the divisibility relation. Thus there is exactly one morphism $d \to m$ in this category if and only if $d$ divides $m$ without remainder; there is no morphism between $d$ and $m$ otherwise. Show that this category has products and coproducts. What are their `conventional' names? [\S VII.5.1]
    \item Redo Exercise 2.9, this time using Proposition 5.4.
    \item Show that in every category $\mathsf{C}$ the products $A \times B$ and $B \times A$ are isomorphic, if they exist. (Hint: Observe that they both satisfy the universal property for the product of $A$ and $B$; then use Proposition 5.4.)
    \item Let $\mathsf{C}$ be a category with products. Find a reasonable candidate for the universal property that the product $A \times B \times C$ of three objects of $\mathsf{C}$ ought to satisfy, and prove that both $(A \times B) \times C$ and $A \times (B \times C)$ satisfy this universal property. Deduce that $(A \times B) \times C$ and $A \times (B \times C)$ are necessarily isomorphic.
    \item Push the envelope a little further still, and define products and coproducts for \textit{families} (i.e., indexed sets) of objects of a category.

          Do these exist in $\mathsf{Set}$?

          It is common to denote the product $\underbrace{A \times \cdots \times A}_{n \text{ times}}$ by $A^n$.

    \item Let $A$, resp. $B$ be a set, endowed with an equivalence relation $\sim_A$, resp. $\sim_B$. Define a relation $\sim$ on $A \times B$ by setting
          \[ (a_1, b_1) \sim (a_2, b_2) \iff a_1 \sim_A a_2 \text{ and } b_1 \sim_B b_2. \]
          (This is immediately seen to be an equivalence relation.)
          \begin{itemize}
            \item Use the universal property for quotients (\S5.3) to establish that there are canonical quotient maps $q_A : A \to A/{\sim_A}$, $q_B : B \to B/{\sim_B}$, and $q : A \times B \to (A \times B)/{\sim_{A \times B}}$, and that these induce functions $(A \times B)/{\sim_{A \times B}} \to A/{\sim_A}$ and $(A \times B)/{\sim_{A \times B}} \to B/{\sim_B}$.
            \item Prove that $(A \times B)/{\sim_{A \times B}}$, together with these induced functions, satisfies the universal property for the product of $A/{\sim_A}$ and $B/{\sim_B}$.
            \item Conclude (without further work) that $(A \times B)/{\sim_{A \times B}} \cong (A/{\sim_A}) \times (B/{\sim_B})$.
          \end{itemize}

    \item Define the notions of \textit{fibered products} and \textit{fibered coproducts}, as terminal objects of the categories $\mathsf{C}_{\alpha,\beta}$, $\mathsf{C}^{\alpha,\beta}$ considered in Example 3.10 (cf. also Exercise 3.11), by stating carefully the corresponding universal properties.

          As it happens, $\mathsf{Set}$ has both fibered products and coproducts. Define these objects `concretely', in terms of naive set theory. [II.3.9, III.6.10, III.6.11]
\end{enumerate}