\section{Functions between sets}
\begin{enumerate}
    \item How many different bijections are there between a set $S$ with $n$ elements and itself? [\S II.2.1]
    \begin{solution}
        Answer: $n!$
        Basic counting
    \end{solution}
    \item Prove statement (2) in Proposition 2.1. You may assume that given a family of disjoint nonempty subsets of a set, there is a way to choose one element in each member of the family\textsuperscript{13}. [\S2.5, V.3.3]
    \begin{solution}
        We first prove that if $f$ has a right inverse $g$, it must be surjective. 
        \\
        For any $x \in B$, $fg(x)=x$. Thus, the image of $f$ is $B$.
        \\
        Next, we prove that if $f$ is surjective, then it must have a right inverse. 
        \\
        We define a ralation on set $A$ such that for any $x,y \in A$, $x\sim y$ if and only if $f(a)=f(b)$. 
        \\
        Let $\mathcal(P)_{\sim}$ be the partition of this relation. By assumption, we can pick one element from each set of $\mathcal(P)_{\sim}$, say $x_1$ to $x_k$. 
        $x_1$ to $x_k$ each corresponding (one to one) to a value in the image of $f$, say $y_1$ to $y_k$.
        \\Thus, we take $g(y_1)=x_1$. 
        It's easy to verify that $g$ is the right inverse of $f$. 
    \end{solution}
    \item Prove that the inverse of a bijection is a bijection and that the composition of two bijections is a bijection.
        \begin{solution}
            \\
            $f$ and $g$ are two bijective functions. They must be inverses $f^{-1}$ and $g^{-1}$. 
            \\
            $f^{-1}$ have right and lef inverse $f$, so it must be a bijection. 
            \\
            $fg$ must be right and left inverse $g^{-1}f^{-1}$, so it must be a bijection. 
        \end[solution]
    \item Prove that `isomorphism' is an equivalence relation (on any set of sets). [\S4.1]
        \begin{solution}
            Refelctive: identity function must exists in $Hom(A,A)$
            \\
            Symmetric: isomorphism must have inverse by definition 4.1.
            \\
            transitive: similar to the proof in 2.3. 
        \end{solution}
    \item Formulate a notion of \textit{epimorphism}, in the style of the notion of \textit{monomorphism} seen in \S2.6, and prove a result analogous to Proposition 2.3, for epimorphisms and surjections. [\S2.6, \S4.2]
        
          \begin{solution}
              The dual notion is as follows: an \textit{epimorphism} is a function $f: A \to B$ such that for any two functions $\alpha', \alpha'': B \to Z$, if $\alpha' \circ f = \alpha'' \circ f$, then $\alpha' = \alpha''$.
              The result analogous to Proposition 2.3 is that $f: A \to B$ is an epimorphism if and only if it is surjective.

              First assume that $f$ is surjective and $\alpha' \circ f = \alpha'' \circ f$. We must show that for any $b \in B$, $\alpha'(b) = \alpha''(b)$. By surjectivity there exists $a \in A$ such that $f(a) = b$. Then $\alpha'(b) = \alpha'(f(a)) = \alpha''(f(a)) = \alpha''(b)$.

              Conversely, assume that $f$ is \emph{not} surjective, so that $\im f$ is not the whole of $B$. Then we can define functions $\alpha',\alpha'': B \to \{0,1\}$ that agree on $\im f$ but differ on some $b \in B \setminus \im f$. Specifically, let $\alpha'(b) = 1$ and $\alpha''(b) = 0$, while $\alpha'$ and $\alpha''$ agree on all other points in $\im f$. Then $\alpha' \circ f = \alpha'' \circ f$, but $\alpha' \neq \alpha''$, showing that $f$ is not an epimorphism.
          \end{solution}

    \item With notation as in Example 2.4, explain how any function $f: A \to B$ determines a section of $\pi_A$.

    \item Let $f: A \to B$ be any function. Prove that the graph $\Gamma_f$ of $f$ is isomorphic to $A$.

    \item Describe as explicitly as you can all terms in the canonical decomposition (cf. \S2.8) of the function $\mathbb{R} \to \mathbb{C}$ defined by $r \mapsto e^{2\pi i r}$. (This exercise matches one assigned previously. Which one?)

    \item Show that if $A' \cong A''$ and $B' \cong B''$, and further $A' \cap B' = \emptyset$ and $A'' \cap B'' = \emptyset$, then $A' \cup B' \cong A'' \cup B''$. Conclude that the operation $\coprod B$ (as described in \S1.4) is well-defined up to isomorphism (cf. \S2.9). [\S2.9, 5.7]

    \item Show that if $A$ and $B$ are finite sets, then $|B^A| = |B|^{|A|}$. [\S2.1, 2.11, \S II.4.1]

          \begin{solution}
              For any $x\in A$, $f(x)$ has $|B|$ possible outputs. Thus, $|B^A|$=$|B|^{|A|}$, by the fundamental principle of counting.
          \end{solution}

    \item In view of Exercise 2.10, it is not unreasonable to use $2^A$ to denote the set of functions from an arbitrary set $A$ to a set with 2 elements (say $\{0,1\}$). Prove that there is a bijection between $2^A$ and the \textit{power set} of $A$ (cf. \S1.2). [\S1.2, III.2.3]

\end{enumerate}