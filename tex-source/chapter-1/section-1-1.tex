\section{Naive set theory}
\begin{enumerate}
    \item Locate a discussion of Russell's paradox, and understand it.

    \item Prove that if $\sim$ is an equivalence relation on a set $S$, then the corresponding family $\mathcal{P}_{\sim}$ defined in \S1.5 is indeed a partition of $S$: that is, its elements are nonempty, disjoint, and their union is $S$. [\S1.5]
        \begin{solution}
            Nonempty is trivial
            Disjoint: For any two elements $A$ and $B$ in $\mathcal{P}_{\sim}$, assume there $x\in A\cap B $, then all elemets in $A$ and $B$ must be equivlent to $x$. Thus, $A$ and $B$ are the same element in $\mathcal{P}_{\sim}$. 
            Union: Since all elements in $\mathcal{P}_{\sim}$ are disjoint and for any element $x$ in $S$, $x$ must be in $[x]_{\sim}$, so theie union must be $S$. 
        \end{solution}
    \item Given a partition $\mathcal{P}$ on a set $S$, show how to define an equivalence relation $\sim$ on $S$ such that $\mathcal{P}$ is the corresponding partition. [\S1.5]
        \begin{solution} 
            For any $X \in \mathcal{P}$, for any $a$ and $b$ in $X$, $a\sim b$. 
            For any two distinct element $X$ and $Y$ in $\mathcal{P}$, for any $x\in X$ and $y\in Y$, $x$ is not equivlent to $y$. 
        \end{solution}
    \item How many different equivalence relations may be defined on the set $\{1, 2, 3\}$?
        \begin{solution}
                Answer: 5
                ${{1,2},{3}}$, ${{1}, {2}, {3}}$, ${{1,3},{2}}$, ${{1},{2,3}}$, ${1,2,3}$
        \end{solution}
    \item Give an example of a relation that is reflexive and symmetric but not transitive. What happens if you attempt to use this relation to define a partition on the set? (Hint: Thinking about the second question will help you answer the first one.)
    \begin{solution}
            Define $S={a,b,c}={2-\sqrt{2},2-\sqrt{3},\sqrt{3}-2}$. For any $a,b \in S$, $a\sim b$ iff $a+b$ is an irrational number. 
            Reflective, for any $a$ in $S$, $a$ is irrational. Thus, $2a$ must be irrational. 
            Symmetric, trivial. 
            Not transitive. $a\sim b$ and $a\sim c$. However, $b+c =0$ is a rational.
    \end[solution]
    \begin{solution}
        Consider, on the set
        \[ S = \{1, \, 2, \, 3\} \]  
        the relation
        \[
            R = \{(1, \, 1), \, (2, \, 2), \, (3, \, 3), \, (1, \, 2), \, (2, \, 1), \, (2, \, 3), \, (3, \, 2)\}.
        \]
        Note that while $R$ is obviously reflexive and symmetric, it is not transitive. (Since $1R2$ and $2R3$, but $1 \cancel{R} 3.$)

        The 'equivalence classes' of $R$ are not disjoint, so a partition of $S$ is not possible.
    \end{solution}

    \item Define a relation $\sim$ on the set $\mathbb{R}$ of real numbers by setting $a \sim b \iff b-a \in \mathbb{Z}$. Prove that this is an equivalence relation, and find a `compelling' description for $\mathbb{R}/\sim$. Do the same for the relation $\approx$ on the plane $\mathbb{R} \times \mathbb{R}$ defined by declaring $(a_1, a_2) \approx (b_1, b_2) \iff b_1 - a_1 \in \mathbb{Z} \text{ and } b_2 - a_2 \in \mathbb{Z}$. [\S II.8.1, II.8.10]
\end{enumerate}
    \begin{solution}
        $\mathbb{R}/\sim ={[a]_{\sim}, a\in [0,1)}$
        $\mathbb{R}^2/\sim = {([a]_{\sim},[b]_\{sim\}), (a,b)\in [0,1)^2}$
    \end{solution}
