\section{Naive set theory}
\begin{enumerate}
    \item Locate a discussion of Russell's paradox, and understand it.
    \item Prove that if $\sim$ is an equivalence relation on a set $S$, then the corresponding family $\mathcal{P}_{\sim}$ defined in \S1.5 is indeed a partition of $S$: that is, its elements are nonempty, disjoint, and their union is $S$. [\S1.5]

          \begin{solution}
            \lipsum[1]

            small test 3
          \end{solution}

    \item Given a partition $\mathcal{P}$ on a set $S$, show how to define a relation $\sim$ on $S$ such that $\mathcal{P}$ is the corresponding partition. [\S1.5]
    \item How many different equivalence relations may be defined on the set $\{1, 2, 3\}$?
    \item Give an example of a relation that is reflexive and symmetric but not transitive. What happens if you attempt to use this relation to define a partition on the set? (Hint: Thinking about the second question will help you answer the first one.)
    \item Define a relation $\sim$ on the set $\mathbb{R}$ of real numbers by setting $a \sim b \iff b-a \in \mathbb{Z}$. Prove that this is an equivalence relation, and find a `compelling' description for $\mathbb{R}/\sim$. Do the same for the relation $\approx$ on the plane $\mathbb{R} \times \mathbb{R}$ defined by declaring $(a_1, a_2) \approx (b_1, b_2) \iff b_1 - a_1 \in \mathbb{Z} \text{ and } b_2 - a_2 \in \mathbb{Z}$. [\S II.8.1, II.8.10]
\end{enumerate}
