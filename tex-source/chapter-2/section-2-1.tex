\section{Definition of group}
\begin{enumerate}
    \item Write a careful proof that every group is the group of isomorphisms of a groupoid. In particular, every group is the group of automorphisms of some object in some category. [\S2.1]

          \begin{solution}
              Let $G$ be a group. The corresponding category $\mathsf{G}$ consists of a single element $*$, and we define $\hom_\mathsf{G}(*,*) = G$. The composition of morphisms is given by the group operation, and the identity morphism is the identity element of $G$. This really is a category since:
              \begin{itemize}
                  \item The composition of morphisms is associative, since the operation of $G$ is associative.
                  \item The identity morphism is the identity element of $G$, which acts as the identity for composition.
              \end{itemize}
              In fact this is even a groupoid, since every morphism is invertible (again by definition of a group). Finally, what are the isomorphisms of this groupoid $\mathsf{G}$? They are precisely the morphisms $\hom_\mathsf{G}(*,*)$, which is $G$ itself. Thus, every group is indeed the group of isomorphisms of a groupoid.
          \end{solution}

    \item Consider the `sets of numbers' listed in \S1.1, and decide which are made into groups by conventional operations such as $+$ and $\cdot$. Even if the answer is negative (for example, $(\mathbb{R}, \cdot)$ is not a group), see if variations on the definition of these sets lead to groups (for example, $(\mathbb{R}^*,\cdot)$ is a group; cf. \S1.4). [\S1.2]

    \item Prove that $(gh)^{-1} = h^{-1}g^{-1}$ for all elements $g, h$ of a group $G$.

    \item Suppose that $g^2 = e$ for all elements $g$ of a group $G$; prove that $G$ is commutative.

    \item The `multiplication table' of a group is an array compiling the results of all multiplications $g \bullet h$:
          \[
              \begin{array}{|c||c|c|c|}
                  \hline
                  \bullet & e      & h           & \cdots \\
                  \hline\hline
                  e       & e      & h           & \cdots \\
                  \hline
                  g       & g      & g \bullet h & \cdots \\
                  \hline
                  \vdots  & \vdots & \vdots      & \ddots \\
                  \hline
              \end{array}
          \]
          (Here $e$ is the identity element. Of course the table depends on the order in which the elements are listed in the top row and leftmost column.) Prove that every row and every column of the multiplication table of a group contains all elements of the group exactly once (like Sudoku diagrams!).

    \item Prove that there is only one possible multiplication table for $G$ if $G$ has exactly 1, 2, or 3 elements. Analyze the possible multiplication tables for groups with exactly 4 elements, and show that there are \textit{two} distinct tables, up to reordering the elements of $G$. Use these tables to prove that all groups with $\le 4$ elements are commutative.

          (You are welcome to analyze groups with 5 elements using the same technique, but you will soon know enough about groups to be able to avoid such brute-force approaches.) [2.19]

    \item Prove Corollary 1.11.

          \begin{solution}
              We must prove that $N \in \mathbb{Z}$ is a multiple of $|g|$ if and only if $g^N = e$.

              For the "only if" direction, $N = k|g|$ for some $k \in \mathbb{Z}$, so \[ g^N = g^{k|g|} = (g^{|g|})^k = e^k = e. \]
              For the converse, we have $e = g^N = (g^{|N|})^{\pm1}$, which implies $e = g^{|N|}$. Then, Lemma~1.10 applies and tells us that $|g|$ divides $|N|$. This means that $\pm N = |N| = k|g|$ for some integer $k$, thus $N = \pm k|g|$ is a multiple of $|g|$.
          \end{solution}

    \item Let $G$ be a finite abelian group with exactly one element $f$ of order 2. Prove that $\prod_{g \in G} g = f$. [4.16]

    \item Let $G$ be a finite group, of order $n$, and let $m$ be the number of elements $g \in G$ of order exactly 2. Prove that $n-m$ is odd. Deduce that if $n$ is even, then $G$ necessarily contains elements of order 2.

    \item Suppose the order of $g$ is odd. What can you say about the order of $g^2$?

    \item Prove that for all $g, h$ in a group $G$, $|gh| = |hg|$. (Hint: Prove that $|aga^{-1}| = |g|$ for all $a, g$ in $G$.)

    \item In the group of invertible $2 \times 2$ matrices, consider
          \[
              g = \begin{pmatrix} 0 & -1 \\ 1 & 0 \end{pmatrix}, \quad h = \begin{pmatrix} 0 & 1 \\ -1 & 0 \end{pmatrix}.
          \]
          Verify that $|g|=4$, $|h|=3$, and $|gh|=\infty$. [\S1.6]

    \item Give an example showing that $|gh|$ is not necessarily equal to $\text{lcm}(|g|,|h|)$, even if $g$ and $h$ commute. [\S1.6, 1.14]

    \item As a counterpoint to Exercise 1.13, prove that if $g$ and $h$ commute and $\text{gcd}(|g|, |h|) = 1$, then $|gh| = |g||h|$. (Hint: Let $N = |g||h|$; then $g^N = (h^{-1})^N$. What can you say about this element?) [\S1.6, 1.15, \S IV.2.5]

    \item Let $G$ be a commutative group, and let $g \in G$ be an element of maximal finite order, that is, such that if $h \in G$ has finite order, then $|h| \le |g|$. Prove that in fact if $h$ has finite order in $G$, then $|h|$ divides $|g|$. (Hint: Argue by contradiction. If $|h|$ is finite but does not divide $|g|$, then there is a prime integer $p$ such that $|g| = p^m r$, $|h| = p^s s$, with $r$ and $s$ relatively prime to $p$ and $m < n$. Use Exercise 1.14 to compute the order of $g^p h^s$.) [\S2.1, 4.11, IV.6.15]
\end{enumerate}