\section{The category Grp}
\begin{enumerate}
    \item Let $\varphi: G \to H$ be a morphism in a category $\mathsf{C}$ with products. Explain why there is a unique morphism $(\varphi \times \varphi): G \times G \to H \times H$ compatible in the obvious way with the natural projections.

          (This morphism is defined explicitly for $\mathsf{C} = \mathsf{Set}$ in \S3.1.) [\S3.1, 3.2]

    \item Let $\varphi: G \to H$, $\psi: H \to K$ be morphisms in a category with products, and consider morphisms between the products $G \times G$, $H \times H$, $K \times K$ as in Exercise 3.1. Prove that
          \[ (\psi\varphi) \times (\psi\varphi) = (\psi \times \psi)(\varphi \times \varphi). \]
          (This is part of the commutativity of the diagram displayed in \S3.2.)

    \item Show that if $G, H$ are abelian groups, then $G \times H$ satisfies the universal property for coproducts in $\mathsf{Ab}$ (cf. \S I.5.5). [\S3.5, 3.6, \S III.6.1]

    \item Let $G, H$ be groups, and assume that $G \cong H \times G$. Can you conclude that $H$ is trivial? (Hint: No. Can you construct a counterexample?)

    \item Prove that $\mathbb{Q}$ is not the direct product of two nontrivial groups.
    \begin{solution}
      If $\mathbb{Q}$ were such a direct product, there would be a projection homomorphism $r:\mathbb{Q} \to Q$ onto a proper subgroup $Q$ such that $r(q) = q$ for all  $q \in Q$ (a so-called \emph{retraction}). I claim no such retraction exists. By the nature of rational numbers and subgroups, there exists some $\frac1p \notin Q$ and also some nonzero integer $n \in Q$. But then \[ np r\left(\frac1p\right) = r(n) = n \implies \frac1p = r\left(\frac1p\right) \in Q,\] a contradiction.
    \end{solution}

    \item Consider the product of the cyclic groups $C_2, C_3$ (cf. \S2.3): $C_2 \times C_3$. By Exercise 3.3, this group is a coproduct of $C_2$ and $C_3$ in $\mathsf{Ab}$. Show that it is not a coproduct of $C_2$ and $C_3$ in $\mathsf{Grp}$, as follows:
          \begin{itemize}
              \item find injective homomorphisms $C_2 \to S_3$, $C_3 \to S_3$;
              \item arguing by contradiction, assume that $C_2 \times C_3$ is a coproduct of $C_2, C_3$, and deduce that there would be a group homomorphism $C_2 \times C_3 \to S_3$ with certain properties;
              \item show that there is no such homomorphism.
          \end{itemize}
          [\S3.5]

    \item Show that there is a surjective homomorphism $\mathbb{Z} \ast \mathbb{Z} \to C_2 \ast C_3$. ($\ast$ denotes coproduct in $\mathsf{Grp}$; cf. \S3.4.)

          One can think of $\mathbb{Z} \ast \mathbb{Z}$ as a group with two generators $x, y$, subject to no relations whatsoever. (We will study a general version of such groups in \S5; see Exercise 5.6.)

    \item Define a group $G$ with two generators $x, y$, subject (only) to the relations $x^2 = e_G$, $y^3 = e_G$. Prove that $G$ is a coproduct of $C_2$ and $C_3$ in $\mathsf{Grp}$. (The reader will obtain an even more concrete description for $C_2 \ast C_3$ in Exercise 9.14; it is called the \textit{modular group}.) [\S3.4, 9.14]

    \item Show that fiber products and coproducts exist in $\mathsf{Ab}$. (Cf. Exercise I.5.12. For coproducts, you may have to wait until you know about quotients.)
\end{enumerate}