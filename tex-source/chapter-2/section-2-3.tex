\section{The category Grp}
\begin{enumerate}
      \item Let $\varphi: G \to H$ be a morphism in a category $\mathsf{C}$ with products. Explain why there is a unique morphism $(\varphi \times \varphi): G \times G \to H \times H$ compatible in the obvious way with the natural projections.

            (This morphism is defined explicitly for $\mathsf{C} = \mathsf{Set}$ in \S3.1.) [\S3.1, 3.2]

      \item Let $\varphi: G \to H$, $\psi: H \to K$ be morphisms in a category with products, and consider morphisms between the products $G \times G$, $H \times H$, $K \times K$ as in Exercise 3.1. Prove that
            \[ (\psi\varphi) \times (\psi\varphi) = (\psi \times \psi)(\varphi \times \varphi). \]
            (This is part of the commutativity of the diagram displayed in \S3.2.)

      \item Show that if $G, H$ are abelian groups, then $G \times H$ satisfies the universal property for coproducts in $\mathsf{Ab}$ (cf. \S I.5.5). [\S3.5, 3.6, \S III.6.1]

      \item Let $G, H$ be groups, and assume that $G \cong H \times G$. Can you conclude that $H$ is trivial? (Hint: No. Can you construct a counterexample?)

      \item Prove that $\mathbb{Q}$ is not the direct product of two nontrivial groups.
            \begin{solution}
                  If $\mathbb{Q}$ were such a direct product, there would be a projection homomorphism $r:\mathbb{Q} \to Q$ onto a proper subgroup $Q$ such that $r(q) = q$ for all  $q \in Q$ (a so-called \emph{retraction}). I claim no such retraction exists. By the nature of rational numbers and subgroups, there exists some $\frac1p \notin Q$ and also some nonzero integer $n \in Q$. But then \[ np r\left(\frac1p\right) = r(n) = n \implies \frac1p = r\left(\frac1p\right) \in Q,\] a contradiction.
            \end{solution}

      \item Consider the product of the cyclic groups $C_2, C_3$ (cf. \S2.3): $C_2 \times C_3$. By Exercise 3.3, this group is a coproduct of $C_2$ and $C_3$ in $\mathsf{Ab}$. Show that it is not a coproduct of $C_2$ and $C_3$ in $\mathsf{Grp}$, as follows:
            \begin{itemize}
                  \item find injective homomorphisms $C_2 \to S_3$, $C_3 \to S_3$;
                  \item arguing by contradiction, assume that $C_2 \times C_3$ is a coproduct of $C_2, C_3$, and deduce that there would be a group homomorphism $C_2 \times C_3 \to S_3$ with certain properties;
                  \item show that there is no such homomorphism.
            \end{itemize}
            [\S3.5]

            \begin{solution}
                  Write $C_2 = \{0,1\}$, $C_3 = \{-1,0,1 \}$. Let $f_1:C_2 \to S_3$ map the generator 1 to the transposition $(1 \, 2)$ and let $f_2:~C_3 \to S_3$ map the generator 1 to the cycle $(1 \, 2 \, 3)$. From the universal property of the coproduct, we would have morphisms $i_1:C_2 \to C_2 \times C_3$, $i_2:C_3 \to C_2 \times C_3$ and a group homomorphism \[ f:C_2 \times C_3 \to S_3 \] such that $f \circ i_1 = f_1$ and $f \circ i_2 = f_2$. Order considerations force $i_1(1) = (1,0)$ and $i_2(1) = (0,\pm1)$, so we have \[ f(1,0) = f_1(1) = (1 \, 2), \quad f(0,1) = f_2(\pm1) = (1 \, 2 \, 3)^{\pm 1}. \]
                  Now, if $i_2(1) = (0,1)$, on one hand \[ f(1,1) = f(1,0)f(0,1) = (1 \, 2)(1 \, 2 \, 3) = (2 \, 3), \] and on the other hand \[ f(1,1) = f(0,1)f(1,0) = (1 \, 2 \, 3)(1 \, 2) = (1 \, 3). \]

                  Similarly, if $i_2(1) = (0,-1)$, we have \[ f(1,1) = f(1,0)f(0,1) = (1 \, 2)(1 \, 3 \, 2) = (1 \, 3), \] and \[ f(1,1) = f(0,1)f(1,0) = (1 \, 3 \, 2)(1 \, 2) = (2 \, 3). \] Either way, we have a contradiction.
            \end{solution}

      \item Show that there is a surjective homomorphism $\mathbb{Z} \ast \mathbb{Z} \to C_2 \ast C_3$. ($\ast$ denotes coproduct in $\mathsf{Grp}$; cf. \S3.4.)

            One can think of $\mathbb{Z} \ast \mathbb{Z}$ as a group with two generators $x, y$, subject to no relations whatsoever. (We will study a general version of such groups in \S5; see Exercise 5.6.)

            \begin{solution}
                  Write the generators of $C_2 \ast C_3$ as $a, b$, where $a$ has order 2 and $b$ has order 3. There are homomorphisms $\mathbb{Z} \rightrightarrows C_2 \ast C_3$ given by sending 1 to $a$ and 1 to $b$. Hence by the universal property of the coproduct, there is a homomorphism $\mathbb{Z} \ast \mathbb{Z} \to C_2 \ast C_3$ given by sending $x$ to $a$ and $y$ to $b$ (identifying $x$ with the generator of the first copy of $\mathbb{Z}$ and $y$ with the generator of the second copy). This homomorphism is surjective because $a$ and $b$ generate $C_2 \ast C_3$.
            \end{solution}

      \item Define a group $G$ with two generators $x, y$, subject (only) to the relations $x^2 = e_G$, $y^3 = e_G$. Prove that $G$ is a coproduct of $C_2$ and $C_3$ in $\mathsf{Grp}$. (The reader will obtain an even more concrete description for $C_2 \ast C_3$ in Exercise 9.14; it is called the \textit{modular group}.) [\S3.4, 9.14]
      
      \begin{solution}
            The canonical homomorphisms are $i_1:C_2 \to C_2 \ast C_3$ and $i_2:C_3 \to C_2 \ast C_3$, sending the generators $1$ to $x$ and $y$, respectively (the homomorphism condition is ensured by the given relations). For the universal property, assume $f_1:C_2 \to H$, $f_2:C_3 \to H$ are homomorphisms into some group $H$. We need to construct a unique homomorphism $f:C_2 \ast C_3 \to H$ such that $f \circ i_1 = f_1$ and $f \circ i_2 = f_2$. Note that $f_1(1)$ must be an element of order 2 in $H$, and $f_2(1)$ must be an element of order 3 in $H$. Thus we can (and must) define $f$ on the generators by setting $f(x) = f_1(1)$ and $f(y) = f_2(1)$.
      \end{solution}

      \item Show that fiber products and coproducts exist in $\mathsf{Ab}$. (Cf. Exercise I.5.12. For coproducts, you may have to wait until you know about quotients.)
\end{enumerate}