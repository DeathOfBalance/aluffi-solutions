\section{Quotient groups}
\begin{enumerate}
    \item List all subgroups of $S_3$ (cf. Exercise 6.13) and determine which subgroups are normal and which are not normal. [\S7.1]

    \item Is the \emph{image} of a group homomorphism necessarily a \emph{normal} subgroup of the target?

    \item Verify that the equivalent conditions for normality given in \S7.1 are indeed equivalent. [\S7.1]

    \item Prove that the relation defined in Exercise 5.10 on a free abelian group $F = F^{\text{ab}}(A)$ is compatible with the group structure. Determine the quotient $F/\sim$ as a better known group.

    \item Define an equivalence relation $\sim$ on $\text{SL}_2(\mathbb{Z})$ by letting $A \sim A' \iff A' = \pm A$. Prove that $\sim$ is compatible with the group structure. The quotient $\text{SL}_2(\mathbb{Z})/\sim$ is denoted $\text{PSL}_2(\mathbb{Z})$ and is called the \textit{modular group}; it is a serious contender in a contest for `the most important group in mathematics', due to its role in algebraic geometry and number theory. Prove that $\text{PSL}_2(\mathbb{Z})$ is generated by the (cosets of the) matrices
          \[ \begin{pmatrix} 0 & -1 \\ 1 & 0 \end{pmatrix} \quad \text{and} \quad \begin{pmatrix} 1 & -1 \\ 1 & 0 \end{pmatrix}. \]
          (You will not need to work very hard, if you use the result of Exercise 6.10.) Note that the first has order 2 in $\text{PSL}_2(\mathbb{Z})$, the second has order 3, and their product has infinite order. [9.14]

    \item Let $G$ be a group, and let $n$ be a positive integer. Consider the relation
          \[ a \sim b \iff (\exists g \in G) \,\, ab^{-1} = g^n. \]
          \begin{itemize}
              \item Show that in general $\sim$ is not an equivalence relation.
              \item Prove that $\sim$ is an equivalence relation if $G$ is commutative, and determine the corresponding subgroup of $G$.
          \end{itemize}

    \item Let $G$ be a group, $n$ a positive integer, and let $H \subseteq G$ be the subgroup generated by all elements of order $n$ in $G$. Prove that $H$ is normal.

    \item Prove Proposition 7.6. [\S7.3]

    \item State and prove the `mirror' statements of Propositions 7.4 and 7.6, leading to the description of relations satisfying ($\dagger\dagger$).

    \item Let $G$ be a group, and $H \subseteq G$ a subgroup. With notation as in Exercise 6.7, show that $H$ is normal in $G$ if and only if $\forall \gamma \in \text{Inn}(G), \gamma(H) \subseteq H$.

          Conclude that if $H$ is normal in $G$, then there is an interesting homomorphism $\text{Inn}(G) \to \text{Aut}(H)$. [8.25]

    \item Let $G$ be a group, and let $[G,G]$ be the subgroup of $G$ generated by all elements of the form $aba^{-1}b^{-1}$. (This is the \textit{commutator subgroup} of $G$; we will return to it in \S IV.3.3.) Prove that $[G,G]$ is normal in $G$. (Hint: With notation as in Exercise 4.8, $g \cdot aba^{-1}b^{-1} \cdot g^{-1} = \gamma(aba^{-1}b^{-1})$. Prove that $G/[G,G]$ is commutative. [7.12, \S IV.3.3]

    \item Let $F = F(A)$ be a free group, and let $f: A \to G$ be a set-function from the set $A$ to a \emph{commutative} group $G$. Prove that $F$ induces a unique homomorphism $F/[F,F] \to G$, where $[F,F]$ is the commutator subgroup of $F$ defined in Exercise 7.11. (Use Theorem 7.12.) Conclude that $F/[F,F] \cong F^{\text{ab}}(A)$. (Use Proposition I.5.4.) [\S6.4, 7.13, VI.1.20]

    \item Let $A, B$ be sets and $F(A), F(B)$ the corresponding free groups. Assume $F(A) \cong F(B)$. If $A$ is finite, prove that $B$ is also and $A \cong B$. (Use Exercise 7.12 to upgrade Exercise 5.10.) [5.10, VI.1.20]

    \item Let $G$ be a group. Prove that $\text{Inn}(G)$ is a \emph{normal} subgroup of $\text{Aut}(G)$.
\end{enumerate}