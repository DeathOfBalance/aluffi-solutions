\section{Canonical decomposition and Lagrange's theorem}
\begin{enumerate}
      \item If a group $H$ may be realized as a subgroup of two groups $G_1$ and $G_2$ and if
            \[ \frac{G_1}{H} \cong \frac{G_2}{H}, \]
            does it follow that $G_1 \cong G_2$? Give a proof or a counterexample.

      \item Extend Example 8.6 as follows. Suppose $G$ is a group and $H \subseteq G$ is a subgroup of index 2, that is, such that there are precisely two (say, left-) cosets of $H$ in $G$. Prove that $H$ is normal in $G$. [9.11, IV.1.16]

      \item Prove that every finite group is finitely presented.

      \item Prove that $(a,b \mid a^2, b^2, (ab)^n)$ is a presentation of the dihedral group $D_{2n}$. (Hint: With respect to the generators defined in Exercise 2.5, set $a=x$ and $b=xy$; prove you can get the relations given here from the ones you obtained in Exercise 2.5, and conversely.)

      \item Let $a, b$ be distinct elements of order 2 in a group $G$, and assume that $ab$ has finite order $n \ge 2$. Prove that the subgroup generated by $a$ and $b$ in $G$ is isomorphic to the dihedral group $D_{2n}$. (Use the previous exercise.)

      \item Let $G$ be a group, and let $A$ be a set of generators for $G$; assume $A$ is finite. The corresponding \emph{Cayley graph}\footnote{Warning: This is one of several alternative conventions.} is a directed graph whose set of vertices is in one-to-one correspondence with $G$, and two vertices $g_1, g_2$ are connected by an edge if $g_2 = g_1 a$ for an $a \in A$; this edge may be labeled $a$ and oriented from $g_1$ to $g_2$. For example, the graph drawn in Example 3.3 for the free group $F(\{x,y\})$ on two generators $x, y$ is the corresponding Cayley graph (with the convention that horizontal edges are labeled $x$ and point to the right and vertical edges are labeled $y$ and point up).

            Prove that if a Cayley graph of a group is a tree, then the group is free. Conversely, prove that free groups admit Cayley graphs that are trees. [\S5.3, 9.15]

      \item Let $(A \mid \mathscr{R})$, resp., $(A' \mid \mathscr{R}')$, be a presentation for a group $G$, resp., $G'$ (cf. \S8.2); we may assume that $A, A'$ are disjoint. Prove that the group $G \ast G'$ presented by
            \[ (A \cup A' \mid \mathscr{R} \cup \mathscr{R}') \]
            satisfies the universal property for the coproduct of $G$ and $G'$ in $\mathsf{Grp}$. (Use the universal properties of both free groups and quotients to construct natural homomorphisms $G \to G \ast G'$, $G' \to G \ast G'$.) [\S3.4, \S8.2, 9.14]

      \item (If you know about matrices (cf. Exercise 6.1)). Prove that $\text{SL}_n(\mathbb{R})$ is a \emph{normal subgroup} of $\text{GL}_n(\mathbb{R})$, and `compute' $\text{GL}_n(\mathbb{R})/\text{SL}_n(\mathbb{R})$ as a well-known group. [VI.3.3]

      \item (Ditto.) Prove that $\text{SO}_3(\mathbb{R}) \cong \text{SU}(2)/\{\pm I_2\}$, where $I_2$ is the identity matrix. (Hint: It so happens that every matrix in $\text{SO}_3(\mathbb{R})$ can be written in the form
            \[
                  \begin{pmatrix}
                        a^2+b^2-c^2-d^2 & 2(bc-ad)        & 2(bd+ac)        \\
                        2(bc+ad)        & a^2-b^2+c^2-d^2 & 2(cd-ab)        \\
                        2(bd-ac)        & 2(cd+ab)        & a^2-b^2-c^2+d^2
                  \end{pmatrix}
            \]
            where $a, b, c, d \in \mathbb{R}$ and $a^2+b^2+c^2+d^2=1$. Proving this fact is not hard, but at this stage you will probably find it computationally demanding. Feel free to assume this, and use Exercise 6.3 to construct a surjective homomorphism $\text{SU}(2) \to \text{SO}_3(\mathbb{R})$; compute the kernel of this homomorphism.)

            If you know a little topology, you can now conclude that the fundamental group\footnote{If you really want to believe this fact, remember that $\mathrm{SO}_3(\mathbb{R})$ parametrizes rotations in $\mathbb{R}^3$. Hold a tray with a glass of water on top of your extended right hand. You should be able to rotate the tray clockwise by a full $360^\circ$ without spilling the water, and your muscles will tell you that the corresponding loop in $\mathrm{SO}_3(\mathbb{R})$ is \emph{not} trivial. But then you will be able to rotate the tray \emph{again} a full $360^\circ$ clockwise without spilling any water, taking it back to the original position. Thus, the square of the loop \emph{is} (homotopically) trivial, as it should be if the fundamental group is cyclic of order 2.}
            of $\text{SO}_3(\mathbb{R})$ is $C_2$. [9.1, VI.1.3]

      \item View $\mathbb{Z} \times \mathbb{Z}$ as a subgroup of $\mathbb{R} \times \mathbb{R}$:
            \[
                  \begin{tikzpicture}
                        \draw[->] (-2.5,0) -- (2.5,0);
                        \draw[->] (0,-2.5) -- (0,2.5);
                        \foreach \x in {-2,-1,0,1,2}
                        \foreach \y in {-2,-1,0,1,2}
                        \node[fill,circle,inner sep=1.5pt] at (\x,\y) {};
                  \end{tikzpicture}
            \]
            Describe the quotient
            \[ \frac{\mathbb{R} \times \mathbb{R}}{\mathbb{Z} \times \mathbb{Z}} \]
            in terms analogous to those used in Example 8.7. (Can you `draw a picture' of this group? Cf. Exercise 1.1.6.)

      \item (Notation as in Proposition 8.10.) Prove `by hand' (that is, without invoking universal properties) that $N$ is normal in $G$ if and only if $N/H$ is normal in $G/H$.

      \item (Notation as in Proposition 8.11.) Prove `by hand' (that is, by using Proposition 6.2) that $HK$ is a subgroup of $G$ if $H$ is normal.

      \item Let $G$ be a finite group, and assume $|G|$ is odd. Prove that every element of $G$ is a square. [8.14]

      \item Generalize the result of Exercise 8.13: if $G$ is a group of order $n$ and $k$ is an integer relatively prime to $n$, then the function $G \to G, g \mapsto g^k$ is surjective.

      \item Let $a, n$ be positive integers, with $a>1$. Prove that $n$ divides $\phi(a^n-1)$, where $\phi$ is Euler's $\phi$-function; see Exercise 6.14. (Hint: Example 8.15.)

      \item Generalize Fermat's little theorem to congruences modulo arbitrary (that is, possibly nonprime) integers. Note that it is \emph{not} true that $a^n \equiv a \pmod{n}$ for all $a$ and $n$: for example, $2^4$ is not congruent to $2$ modulo $4$. \emph{What is} true? (This generalization is known as \emph{Euler's theorem}.)

      \item Assume $G$ is a finite abelian group, and let $p$ be a prime divisor of $|G|$. Prove that there exists an element in $G$ of order $p$. (Hint: Let $g \neq e$ be an element of $G$, and consider the subgroup $\langle g \rangle$; use the fact that this subgroup is cyclic to show that there is an element $h \in \langle g \rangle$ of prime order $q$. If $q=p$, you are done; otherwise, use the quotient $G/\langle h \rangle$ and induction.) [\S8.5, 8.18, 8.20, \S IV.2.1]

      \item Let $G$ be an abelian group of order $2n$, where $n$ is odd. Prove that $G$ has \emph{exactly one} element of order 2. (It has at least one, for example by Exercise 8.17. Use Lagrange's theorem to establish that it cannot have more than one.) Does the same conclusion hold if $G$ is not necessarily commutative?

      \item Let $G$ be a finite group, and let $d$ be a proper divisor of $|G|$. Is it necessarily true that there exists an element of $G$ of order $d$? Give a proof or a counterexample.

      \item Assume $G$ is a finite abelian group, and let $d$ be a divisor of $|G|$. Prove that there exists a \emph{subgroup} $H \subseteq G$ of order $d$. (Hint: induction; use Exercise 8.17.) [\S IV.2.2]

      \item Let $H, K$ be subgroups of a group $G$. Construct a bijection between the set of cosets $hK$ with $h \in H$ and the set of left-cosets of $H \cap K$ in $H$. If $H$ and $K$ are finite, prove that
            \[ |HK| = \frac{|H| \cdot |K|}{|H \cap K|}. \]
            [\S8.5, \S IV.4.4]

      \item Let $\varphi: G \to G'$ be a group homomorphism, and let $N$ be the smallest normal subgroup containing $\text{im}\,\varphi$. Prove that $G'/N$ satisfies the universal property of coker $\varphi$ in $\mathsf{Grp}$. [\S8.6]

      \item Consider the subgroup
            \[ H = \left\{ \begin{pmatrix} 1 & 2 & 3 \\ 1 & 2 & 3 \end{pmatrix}, \begin{pmatrix} 1 & 2 & 3 \\ 2 & 1 & 3 \end{pmatrix} \right\} \]
            of $S_3$. Show that the cokernel of the inclusion $H \hookrightarrow S_3$ is trivial, although $H \hookrightarrow S_3$ is not surjective. [\S8.6]

      \item Show that epimorphisms in $\mathsf{Grp}$ do not necessarily have right-inverses. [\S I.4.2]

      \item Let $H$ be a commutative normal subgroup of $G$. Construct an interesting homomorphism from $G/H$ to $\text{Aut}(H)$. (Cf. Exercise 7.10.)
\end{enumerate}