\section{Examples of groups}
\begin{enumerate}
    \item One can associate an $n \times n$ matrix $M_\sigma$ with a permutation $\sigma \in S_n$ by letting the entry at $(i, (i)\sigma)$ be 1 and letting all other entries be 0. For example, the matrix corresponding to the permutation
          \[ \sigma = \begin{pmatrix} 1 & 2 & 3 \\ 3 & 1 & 2 \end{pmatrix} \in S_3 \]
          would be
          \[ M_\sigma = \begin{pmatrix} 0 & 0 & 1 \\ 1 & 0 & 0 \\ 0 & 1 & 0 \end{pmatrix}. \]
          Prove that, with this notation,
          \[ M_{\sigma\tau} = M_\sigma M_\tau \]
          for all $\sigma, \tau \in S_n$, where the product on the right is the ordinary product of matrices. [IV.4.13]

    \item Prove that if $d \le n$, then $S_n$ contains elements of order $d$. [\S2.1]

    \item For every positive integer $n$ find an element of order $n$ in $S_n$.

    \item Define a homomorphism $D_8 \to S_4$ by labeling vertices of a square, as we did for a triangle in \S2.2. List the 8 permutations in the image of this homomorphism.

    \item Describe generators and relations for all dihedral groups $D_{2n}$. (Hint: Let $r$ be the reflection about a line through the center of a regular $n$-gon and a vertex, and let $y$ be the counterclockwise rotation by $2\pi/n$. The group $D_{2n}$ will be generated by $x$ and $y$, subject to three relations\footnotemark. To see that these relations really determine $D_{2n}$, use them to show that any product $x^i y^j x^k y^l \cdots$ equals $x^a y^b$ for some $i, j$ with $0 \le i \le 1$, $0 \le j < n$.) [\S8.4, \S IV.2.5]
        \footnote{Two relations are evident. To ‘see’ the third one, hold your right hand in front of and
        away from you, pointing your fingers at the vertices of an imaginary regular pentagon. Flip the
        pentagon by turning the hand toward you; rotate it counterclockwise w.r.t. the line of sight by $72^\circ$; flip it again by pointing it away from you; and rotate it counterclockwise a second time. This
returns the hand to the initial position. What does this tell you?}

    \item For every positive integer $n$ construct a group containing elements $g, h$ such that $|g|=2$, $|h|=2$, and $|gh|=n$. (Hint: For $n > 1$, $D_{2n}$ will do.) [\S1.6]

    \item Find all elements of $D_{2n}$ that commute with every other element. (The parity of $n$ plays a role.) [IV.1.2]

    \item Find the orders of the groups of symmetries of the five `platonic solids'.

    \item Verify carefully that `congruence mod $n$' is an equivalence relation.

    \item Prove that if $n > 0$, then $\mathbb{Z}/n\mathbb{Z}$ consists of precisely $n$ elements.

    \item Prove that the square of every odd integer is congruent to 1 modulo 8. [\S VII.5.1]

    \item Prove that there are no nonzero integers $a, b, c$ such that $a^2+b^2=3c^2$. (Hint: By studying the equation $a^2+b^2=3c^2$ in $\mathbb{Z}/4\mathbb{Z}$, show that $a, b, c$ would all have to be even. Letting $a=2k$, $b=2f$, $c=2m$, you would have $k^2+f^2=3m^2$. What's wrong with that?)

    \item Prove that if $\text{gcd}(m, n)=1$, then there exist integers $a$ and $b$ such that $am+bn=1$.

          (Use Corollary 2.5.) Conversely, prove that if $am+bn=1$ for some integers $a$ and $b$, then $\text{gcd}(m, n)=1$. [2.15, \S V.2.1, V.2.4]

    \item State and prove an analog of Lemma 2.2, showing that the multiplication on $\mathbb{Z}/n\mathbb{Z}$ is a well-defined operation. [\S2.3, \S III.1.2]

    \item Let $n > 0$ be an odd integer.
          \begin{itemize}
              \item Prove that if $\text{gcd}(m, n)=1$, then $\text{gcd}(2m+n, 2n)=1$. (Use Exercise 2.13.)
              \item Prove that if $\text{gcd}(r, 2n)=1$, then $\text{gcd}(\frac{r-n}{2}, n)=1$. (Ditto.)
              \item Conclude that the function $[m] \mapsto [\frac{2m+n}{2n}]$ is a bijection between $(\mathbb{Z}/n\mathbb{Z})^*$ and $(\mathbb{Z}/2n\mathbb{Z})^*$.
          \end{itemize}
          The number $\phi(n)$ of elements of $(\mathbb{Z}/n\mathbb{Z})^*$ is Euler's $\phi$-function. The reader has just proved that if $n$ is odd, then $\phi(2n) = \phi(n)$. Much more general formulas will be given later on (cf. Exercise V.6.8). [VII.5.11]

    \item Find the last digit of $1238237^{18238456}$. (Work in $\mathbb{Z}/10\mathbb{Z}$.)

    \item Show that if $m \equiv m' \pmod n$, then $\text{gcd}(m, n)=1$ if and only if $\text{gcd}(m', n)=1$. [\S2.3]

    \item For $d \le n$, define an injective function $\mathbb{Z}/d\mathbb{Z} \to S_n$ preserving the operation, that is, such that the sum of equivalence classes in $\mathbb{Z}/d\mathbb{Z}$ corresponds to the product of the corresponding permutations.

    \item Both $(\mathbb{Z}/5\mathbb{Z})^*$ and $(\mathbb{Z}/12\mathbb{Z})^*$ consist of 4 elements. Write their multiplication tables, and prove that no re-ordering of the elements will make them match. (Cf. Exercise 1.6.) [\S4.3]
\end{enumerate}